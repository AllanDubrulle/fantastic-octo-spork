\documentclass[a4paper, 12pt]{article}
% Instructions de compilation:
% Une fois LaTeX, une fois BibTex, deux fois LaTeX
\usepackage[utf8]{inputenc}
\usepackage[T1]{fontenc}
\usepackage[francais]{babel}
\usepackage{amsmath, amssymb, amsthm, verbatim}
\usepackage{IEEEtrantools}
\usepackage[margin=1in]{geometry}
\usepackage[colorlinks, linkcolor=blue]{hyperref}
\usepackage{epigraph}
\usepackage{mathrsfs}
\usepackage{algorithm}
\usepackage{algorithmic}
\usepackage{tikz-cd}
\usepackage{xcolor}
\usepackage[title]{appendix}
\usepackage{tabu} %pour des tableaux plus beaux
\usepackage[skip=0pt]{caption}
% Définition des théorèmes
\theoremstyle{definition} \newtheorem{df}{D\'{e}finition}[section]
\theoremstyle{definition} \newtheorem{ex}[df]{Exemple}
\theoremstyle{definition} \newtheorem{thm}[df]{Th\'{e}or\`{e}me}
\theoremstyle{definition} \newtheorem{cor}[df]{Corollaire}
\theoremstyle{definition} \newtheorem{lem}[df]{Lemme}
\theoremstyle{definition} \newtheorem{obs}[df]{Observation}
\theoremstyle{definition} \newtheorem{prop}[df]{Proposition}
\theoremstyle{definition} \newtheorem{rem}[df]{Remarque}

\title{Projet de Structures de Données II: Rapport Final -- Annexes}
\author{A. Dubrulle, J. Main}
\date{\today}

\begin{document}
\maketitle
\section*{Tableaux de performances}
Ce document reprend l'entièreté des résultats expérimentaux
réalisés en Java comparant les trois heuristiques qui nous ont été assignées
pendant le projet. Toutes les informations relatives à la manière
dont les tests ont été effectués sont consignées dans le rapport final.

\begin{table}[h]
\caption{randomSmall.txt}
\begin{center}
{\tabulinesep=1.2mm
\begin{tabu}{|l|c|c|c|}
  \hline
  & Heuristique linéaire  & Heuristique 1 & Heuristique aléatoire \\ 
  \hline
  Taille &      9079&      6967&    7807,5  \\ 
  \hline
  Hauteur &        55&        26&      30,0  \\ 
  \hline
  Temps constructeur (ms) &          1,7740&        147,5215&          1,1566  \\ 
  \hline
  Temps peintre (ms) &            7,8894&           5,7719&           6,5705  \\ 
  \hline
\end{tabu}
}
\end{center}

%%%Local Variables:
%%% mode: latex
%%% TeX-master: "../rapportGp1"
%%% End:
\end{table}

\begin{table}[h]
\caption{randomMedium.txt}
\begin{center}
{\tabulinesep=1.2mm
\begin{tabu}{|l|c|c|c|}
  \hline
  & Heuristique linéaire  & Heuristique 1 & Heuristique aléatoire \\ 
  \hline
  Taille &     34153&     25559&   27707,5  \\ 
  \hline
  Hauteur &        96&        35&      38,0  \\ 
  \hline
  Temps constructeur (ms) &          8,5829&       1255,3063&         12,1229  \\ 
  \hline
  Temps peintre (ms) &           28,6280&          21,2297&          23,0759  \\ 
  \hline
\end{tabu}
}
\end{center}

%%%Local Variables:
%%% mode: latex
%%% TeX-master: "../rapportGp1"
%%% End:
\end{table}

\begin{table}[h]
\caption{randomLarge.txt}
\begin{center}
{\tabulinesep=1.2mm
\begin{tabu}{|l|c|c|c|}
  \hline
  & Heuristique linéaire  & Heuristique 1 & Heuristique aléatoire \\ 
  \hline
  Taille &     72315&     52971&   56777,0  \\ 
  \hline
  Hauteur &       119&        42&      44,1  \\ 
  \hline
  Temps constructeur (ms) &         20,9287&       7969,2461&         16,4356  \\ 
  \hline
  Temps peintre (ms) &           61,4867&          44,1814&          47,5941  \\ 
  \hline
\end{tabu}
}
\end{center}

%%%Local Variables:
%%% mode: latex
%%% TeX-master: "../rapportGp1"
%%% End:
\end{table}

\begin{table}[h]
\caption{randomHuge.txt}
\begin{center}
{\tabulinesep=1.2mm
\begin{tabu}{|l|c|c|c|}
  \hline
  & Heuristique linéaire  & Heuristique 1 & Heuristique aléatoire \\ 
  \hline
  Taille &    141415&    105197&  111894,9  \\ 
  \hline
  Hauteur &       157&        50&      49,4  \\ 
  \hline
  Temps constructeur (ms) &         53,4573&      25755,7597&         51,0537  \\ 
  \hline
  Temps peintre (ms) &          124,8024&          89,8473&          94,6599  \\ 
  \hline
\end{tabu}
}
\end{center}

%%%Local Variables:
%%% mode: latex
%%% TeX-master: "../rapportGp1"
%%% End:
\end{table}


\begin{table}[h]
\caption{ellipsesSmall.txt}
\begin{center}
{\tabulinesep=1.2mm
\begin{tabu}{|l|c|c|c|}
  \hline
  & Heuristique linéaire  & Heuristique 1 & Heuristique aléatoire \\ 
  \hline
  Taille &       401&       401&     421,7  \\ 
  \hline
  Hauteur &       201&       155&      59,6  \\ 
  \hline
  Temps constructeur (ms) &          0,2428&          7,1014&          0,0603  \\ 
  \hline
  Temps peintre (ms) &            0,3253&           0,3205&           0,3525  \\ 
  \hline
\end{tabu}
}
\end{center}

%%%Local Variables:
%%% mode: latex
%%% TeX-master: "../rapportGp1"
%%% End:
\end{table}

\begin{table}[h]
\caption{ellipsesMedium.txt}
\begin{center}
{\tabulinesep=1.2mm
\begin{tabu}{|l|c|c|c|}
  \hline
  & Heuristique linéaire  & Heuristique 1 & Heuristique aléatoire \\ 
  \hline
  Taille &      1441&      1445&    1493,3  \\ 
  \hline
  Hauteur &       721&       377&     133,4  \\ 
  \hline
  Temps constructeur (ms) &          2,9667&        136,0990&          0,2764  \\ 
  \hline
  Temps peintre (ms) &            1,1727&           1,1599&           1,2801  \\ 
  \hline
\end{tabu}
}
\end{center}

%%%Local Variables:
%%% mode: latex
%%% TeX-master: "../rapportGp1"
%%% End:
\end{table}

\begin{table}[h]
\caption{ellipsesLarge.txt}
\begin{center}
{\tabulinesep=1.2mm
\begin{tabu}{|l|c|c|c|}
  \hline
  & Heuristique linéaire  & Heuristique 1 & Heuristique aléatoire \\ 
  \hline
  Taille &      9001&      9087&    9116,7  \\ 
  \hline
  Hauteur &      4501&      1713&     516,8  \\ 
  \hline
  Temps constructeur (ms) &        114,2485&      15021,3817&          3,4645  \\ 
  \hline
  Temps peintre (ms) &            7,6303&           7,7376&           8,0163  \\ 
  \hline
\end{tabu}
}
\end{center}

%%%Local Variables:
%%% mode: latex
%%% TeX-master: "../rapportGp1"
%%% End:
\end{table}

\begin{table}[h]
\caption{rectanglesSmall.txt}
\begin{center}
{\tabulinesep=1.2mm
\begin{tabu}{|l|c|c|c|}
  \hline
  & Heuristique linéaire  & Heuristique 1 & Heuristique aléatoire \\ 
  \hline
  Taille &        49&        49&     127,2  \\ 
  \hline
  Hauteur &        25&        25&      12,7  \\ 
  \hline
  Temps constructeur (ms) &          0,9088&         19,6237&          0,3830  \\ 
  \hline
  Temps peintre (ms) &            2,7720&           1,2075&           1,2262  \\ 
  \hline
\end{tabu}
}
\end{center}

%%%Local Variables:
%%% mode: latex
%%% TeX-master: "../rapportGp1"
%%% End:
\end{table}

\begin{table}[h]
\caption{rectanglesMedium.txt}
\begin{center}
{\tabulinesep=1.2mm
\begin{tabu}{|l|c|c|c|}
  \hline
  & Heuristique linéaire  & Heuristique 1 & Heuristique aléatoire \\ 
  \hline
  Taille &        41&        45&      93,7  \\ 
  \hline
  Hauteur &        21&        21&      11,3  \\ 
  \hline
  Temps constructeur (ms) &          0,3883&        133,0004&          0,2548  \\ 
  \hline
  Temps peintre (ms) &            3,6132&           3,5464&           3,2909  \\ 
  \hline
\end{tabu}
}
\end{center}

%%%Local Variables:
%%% mode: latex
%%% TeX-master: "../rapportGp1"
%%% End:
\end{table}

\begin{table}[h]
\caption{rectanglesLarge.txt}
\begin{center}
{\tabulinesep=1.2mm
\begin{tabu}{|l|c|c|c|}
  \hline
  & Heuristique linéaire  & Heuristique 1 & Heuristique aléatoire \\ 
  \hline
  Taille &        89&        89&     334,5  \\ 
  \hline
  Hauteur &        45&        45&      15,1  \\ 
  \hline
  Temps constructeur (ms) &          1,7089&       2164,4079&          1,0345  \\ 
  \hline
  Temps peintre (ms) &           11,5583&          11,5322&          11,6792  \\ 
  \hline
\end{tabu}
}
\end{center}

%%%Local Variables:
%%% mode: latex
%%% TeX-master: "../rapportGp1"
%%% End:
\end{table}

\begin{table}[h]
\caption{rectanglesHuge.txt}
\begin{center}
{\tabulinesep=1.2mm
\begin{tabu}{|l|c|c|c|}
  \hline
  & Heuristique linéaire  & Heuristique 1 & Heuristique aléatoire \\ 
  \hline
  Taille &       115&       311&     313,5  \\ 
  \hline
  Hauteur &        46&        24&      15,0  \\ 
  \hline
  Temps constructeur (ms) &          5,2832&       3033,3284&          2,2017  \\ 
  \hline
  Temps peintre (ms) &           27,9096&          29,7665&          26,5652  \\ 
  \hline
\end{tabu}
}
\end{center}

%%%Local Variables:
%%% mode: latex
%%% TeX-master: "../rapportGp1"
%%% End:
\end{table}

\end{document}
%%% Local Variables:
%%% mode: latex
%%% TeX-master: t
%%% End:
