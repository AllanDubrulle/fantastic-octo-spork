Ce projet était un travail enrichissant, étant donné que nous n'avons
pas beaucoup de cours demandant une mise en pratique de la programmation
orienté objet dans notre parcours, même s'il ne s'agissait que d'un
aspect secondaire du travail. La partie la plus compliquée était
sans doute celle liée à fournir une représentation de ce que voit
l'oeil à partir d'un segment.

Les tests effectués démontrent l'utilité de construire les arbres BSP
dans un ordre différent que celui dans lequel les données sont fournies;
les approches $H_1$ et aléatoires fournissent de meilleurs résultats
la plupart du temps. Comme attendu, le coût de construction de l'arbre
est plus élevé dans le cas de l'heuristique $H_1$. Toutefois, certaines
scènes ne voient pas beaucoup d'améliorations avec l'heuristique $H_1$
comparée aux moyennes calculées dans le cas aléatoire.


%%% Local Variables:
%%% mode: latex
%%% TeX-master: "../rapportGp1"
%%% End:
