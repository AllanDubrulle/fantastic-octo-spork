\subsection{Heuristique linéaire}
L'heuristique linéaire consiste simplement à retourner le premier élément de la liste $S$.

\begin{prop}\label{hLin:compl}
 La complexité de l'heuristique linéaire est en $O(1)$.
\end{prop}
\begin{proof}
  La complexité d'obtenir un élément à une position donnée dans une liste est en $O(1)$.  
\end{proof}

\subsection{Heuristique aléatoire}

L'heuristique aléatoire revient à appliquer l'heuristique linéaire à une liste $S'$
où $S'$ est une permutation de $S$. la méthode shuffle de la classe Java.util.Collections permet de faire
une telle permutation et sa complexité est en $O(N)$ où $N$ est la longueur de la liste.
Vu que l'appel à shuffle n'a lieu qu'au premier appel, on a un travail préparation en $O(N)$.

\begin{prop}
 La complexité de l'heuristique aléatoire est  en $O(1)$.
\end{prop}
\begin{proof}
  Voir \ref{hLin:compl}.  
\end{proof}


\subsection{Heuristique H1}
Soient $S$ une liste de segments, $s$ et $s'$ deux segments
de $S$ tels que $s \ne s'$. La droite supportant le segment $s$ (resp. $s'$
) est notée $D_s$ (resp. $D_{s'}$).
Plusieurs configurations sont possibles:
\begin{enumerate}
\item les droites  $D_s$ et  $D_{s'}$ sont confondues; \label{h1:enum1}
\item la droite $D_s$ est parallèle à $D_{s'} $ et elles sont disjointes; \label{h1:enum2}
\item les segments $s$ et $s'$ ont une intersection; \label{h1:enum3}
\item la droite $D_{s'}$ intersecte $s$ en un point
  n'appartenant pas à $s'$;\label{h1:enum4}
\item la droite $D_{s}$ intersecte $s'$ en un point
  n'appartenant pas à $s$; \label{h1:enum5}
\item les droites $D_s$ et $D_{s'}$ s'intersectent en un point n'appartenant ni
  à $s$, ni à $s'$. \label{h1:enum6}
\end{enumerate}

Il est important de remarquer que ces cas sont deux-à-deux disjoints.
En particulier les cas \ref{h1:enum4} et \ref{h1:enum5} sont disjoints.

Soit $g_s$ le nombre de droites $D'$ supportant un segment $s' \neq s$
de $S$ qui intersectent le segment $s$.

Maximiser $g_s$ pour $s \in S$ revient à choisir le segment $s$ qui maximise
le nombre de cas \ref{h1:enum3} et \ref{h1:enum4} pour $s' \in S$.
Par ce choix, le nombre de cas \ref{h1:enum5} est minimisé.
Par conséquent, on peut espérer que la droite $D_s$ intersecte un
nombre minimal de segment $s'$, ce qui équivaut au fait que le
nombre de fragment généré en séparant le plan par $D_s$ sera minimal.

Remarquons que pour cette heuristique, les 2 premiers cas ainsi que
le cas \ref{h1:enum6} ne sont pas exploités.
Le cas \ref{h1:enum3} est un cas particulier pour l'heuristique;
il est intéressant de noter que si deux segments s'intersectent
en un point différent de l'une de leurs extrémités, alors ils devront
éventuellement être fragmentés, et s'ils partagent une extrémité, alors
en choisir la droite supportant un des segments pour séparer le plan ne
fragmentera pas le premier.

Cette heuristique a pour but d'espérer une réduction du nombre de fragments
pour diminuer la taille de l'arbre
et donc améliorer le temps de l'algorithme du peintre qui dépend
fortement de la taille de l'arbre et du nombre de fragments générés (par
le corollaire \ref{peintre:cor}). L'algorithme \ref{algo:h1} détaille
la procédure associée à H1.

\begin{algorithm}
  \caption{H1($S$)}
  \begin{algorithmic}[1] \label{algo:h1}
    \REQUIRE $S$ une liste de segments non vide.
    \ENSURE $s$ tel que $g_s$ est maximal.
    \STATE$r\leftarrow$ le premier segment de $S$
    \FOR{$s \in S$}
    \STATE $g_s\leftarrow 0$
    \ENDFOR
    \FOR{$s \in S$}
    \FOR{$s' \in S$}
    \IF{$s \neq s'$ et  $D_{s'}$ intersecte $s$}
    \STATE $g_s\leftarrow g_s +1$
    \ENDIF
    \ENDFOR
    \IF{$g_s>g_r$}
    \STATE $r\leftarrow s$
    \ENDIF
    \ENDFOR
    \RETURN $r$
  \end{algorithmic}
\end{algorithm}

\begin{prop}
  La complexité de l'algorithme H1 est en $O(N^2)$ où $N =$ la taille de S.
\end{prop}
\begin{proof}
  L'algorithme itère sur chaque segment $s \in S$.
  On réitère ensuite sur chaque segment $s' \in S$.
  L'intérieur des boucles correspond à du calcul et
  est donc en $O(1)$. On obtient que H1 est en $N\times(N\times{}O(1)) = O(N^2)$
\end{proof}



%%% Local Variables:
%%% mode: latex
%%% TeX-master: "../rapportGp1"
%%% End:
