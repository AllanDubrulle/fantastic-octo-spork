\subsection{Heuristiques employées}\label{heur:section}
Les différentes heuristiques employées pour la construction des arbres
BSP sont décrites et leur complexité est étudiée dans cette section.

La lettre $S$ désigne un ensemble non vide de segments. \'Etant donné
un segment $s$ non réduit à un point, $D_s$ dénote la droite
supportant le segment $s$.
\subsubsection*{Heuristique linéaire}
L'heuristique linéaire consiste simplement à retourner le premier
élément de la liste $S$. Ainsi, la liste des segments composant
la scène est parcourue dans l'ordre dans lequel elle est fournie.

\begin{prop}\label{hLin:compl}
 La complexité de l'heuristique linéaire est en $O(1)$.
\end{prop}
\begin{proof}
  La complexité d'obtenir un élément en tête de liste
  est en $O(1)$.
\end{proof}

\subsubsection*{Heuristique aléatoire}

L'heuristique aléatoire revient à appliquer l'heuristique linéaire
à une liste $S'$ où $S'$ est une permutation de $S$. La méthode
shuffle de la classe \texttt{Java.util.Collections}
(employée dans l'implémentation) permet de faire
une telle permutation avec une complexité en $O(N)$
où $N$ est la longueur de la liste.
Vu que l'appel à shuffle a lieu avant l'appel au constructeur de l'arbre,
on a un travail de préparation en $O(N)$.

\begin{prop}
  La complexité de la sélection d'un segment dans
  l'approche aléatoire est en $O(1)$.
\end{prop}
\begin{proof}
  Voir la proposition \ref{hLin:compl}.
\end{proof}


\subsubsection*{Heuristique $H_1$}
Soit $g_s$ le nombre de droites $D'$ supportant un segment $s' \neq s$
de $S$ qui intersectent le segment $s$, pour $s$ un segment de $S$.
L'idée de l'heuristique est de choisir comme droite séparatrice du plan
une droite portant un segment $s$ tel que $g_s$ est maximal.

Soient $s$ et $s'$ deux segments de $S$ tels que $s \ne s'$.
Plusieurs configurations sont possibles:
\begin{enumerate}
\item les droites  $D_s$ et  $D_{s'}$ sont confondues; \label{h1:enum1}
\item la droite $D_s$ est parallèle à $D_{s'} $ et elles sont disjointes; \label{h1:enum2}
\item les segments $s$ et $s'$ ont une intersection; \label{h1:enum3}
\item la droite $D_{s'}$ intersecte $s$ en un point
  n'appartenant pas à $s'$;\label{h1:enum4}
\item la droite $D_{s}$ intersecte $s'$ en un point
  n'appartenant pas à $s$; \label{h1:enum5}
\item les droites $D_s$ et $D_{s'}$ s'intersectent en un point n'appartenant ni
  à $s$, ni à $s'$. \label{h1:enum6}
\end{enumerate}

Il est important de remarquer que ces cas sont deux-à-deux disjoints.
En particulier les cas \ref{h1:enum4} et \ref{h1:enum5} sont disjoints.

Maximiser $g_s$ pour $s \in S$ revient à choisir le segment $s$ qui maximise
le nombre de cas \ref{h1:enum3} et \ref{h1:enum4} pour $s' \in S$.
Par ce choix, on espère que le nombre de cas \ref{h1:enum5} est minimisé.
Si peu de cas \ref{h1:enum3} se produisent, on espère que la droite $D_s$ intersecte un
nombre minimal de segments $s'$, ce qui équivaut au fait que le
nombre de fragments générés en séparant le plan par $D_s$ sera minimal.
 


Il est intéressant de noter que si une droite $D$ contient $s$, alors
en particulier elle intersecte $s$. Ceci implique qu'un segment supportant
une droite contenant beaucoup d'autres segments de la scène sera plus
susceptible d'être choisi, ce qui impactera la taille de l'arbre en construction.

Remarquons que pour cette heuristique, les cas \ref{h1:enum2} et
\ref{h1:enum6} ne sont pas exploités.
Le cas \ref{h1:enum3} est un cas particulier pour l'heuristique;
il est intéressant de noter que si deux segments s'intersectent
en un point différent de l'une de leurs extrémités, alors ils devront
éventuellement être fragmentés, et s'ils partagent une extrémité, alors
en choisir la droite supportant un des segments pour séparer le plan ne
fragmentera pas le premier.

Cette heuristique a pour but d'espérer une réduction du nombre de fragments
pour diminuer la taille de l'arbre
et donc améliorer le temps de l'algorithme du peintre qui dépend
fortement de la taille de l'arbre et du nombre de fragments générés (par
le corollaire \ref{peintre:cor}). L'algorithme \ref{algo:h1} détaille
la procédure associée à $H_1$.

\begin{algorithm}
  \caption{$H_1$($S$)}
  \begin{algorithmic}[1] \label{algo:h1}
    \REQUIRE $S$ une liste de segments non vide.
    \ENSURE $s$ tel que $g_s$ est maximal.
    \STATE$r\leftarrow$ le premier segment de $S$
    \FOR{$s \in S$}
    \STATE $g_s\leftarrow 0$
    \ENDFOR
    \FOR{$s \in S$}
    \FOR{$s' \in S$}
    \IF{$s \neq s'$ et  $D_{s'}$ intersecte $s$}\label{algoh1:inter}
    \STATE $g_s\leftarrow g_s +1$
    \ENDIF
    \ENDFOR
    \IF{$g_s>g_r$}
    \STATE $r\leftarrow s$
    \ENDIF
    \ENDFOR
    \RETURN $r$
  \end{algorithmic}
\end{algorithm}

\begin{prop}
  La complexité de l'algorithme $H_1$ est en $O(N^2)$ où $N$ est la taille de S.
\end{prop}
\begin{proof}
  L'algorithme itère sur chaque segment $s \in S$ une première fois afin
  d'initialiser chaque valeur de $g_s$. Ensuite, l'algorithme itère à nouveau
  sur chaque segment de $S$; cette boucle contient une boucle interne
  itérant sur chaque segment $s' \in S$ et réalisant un travail en $O(1)$
  à chaque itération. En particulier, le coût de l'instruction
  \ref{algoh1:inter} est en $O(1)$; la détermination de la
  droite $D_{s'}$ est en $O(1)$ par le lemme \ref{lem:droite} et
  l'existence d'une intersection est équivalente à avoir une des
  extrémités sur la droite ou chaque extrémité dans un demi-plan
  différent vis-à-vis de la droite considérée.


  En dehors de cette boucle interne se trouve une instruction
  conditionnelle également en temps constant.
  On obtient que $H_1$ est en $N\times(N\times{}O(1) + O(1)) = O(N^2)$
\end{proof}



%%% Local Variables:
%%% mode: latex
%%% TeX-master: "../rapportGp1"
%%% End:
