\subsection{Analyse des temps CPU}
Les tests ont été effectués avec un \oe{}il à la position $(0,0)$ et un
angle de 1 radian. Les données employées pour les tests sont les scènes
qui nous ont été fournies sur la plateforme e-learning.

La configuration du PC employé pour les tests était la suivante :

\begin{center}
  Intel Core i5-7300HQ CPU @ 2.50GHz $\times$ 4,  8Gb ram, système Ubuntu 18.04.2 LTS,
  Java version "1.8.0\_191",
\end{center}
Pour calculer le temps CPU, la méthode \texttt{nanotime} de la classe
\texttt{System} a été utilisée.

Les tables contenant les résultats expérimentaux sont données
à l'annexe \ref{ann:res}.

Par la suite, la formulation \og{}la hauteur de l'heuristique
$X$\fg{} sera abusivement employée pour parler de \og{}la hauteur
de l'arbre générée par l'heuristique $X$\fg{}
(sous-entendu hauteur moyenne dans le cas de l'heuristique aléatoire).
Il en va de même pour les autres caractéristiques.

\subsubsection*{Phénomènes globaux}
Globalement, on peut remarquer que l'heuristique $H_1$ prend beaucoup plus de
temps à construire l'arbre que les autres heuristiques.
Comme attendu par la proposition \ref{compl:painter}, on remarque
que si une heuristique a une taille moins importante qu'une autre
alors l'algorithme du peintre est plus efficace pour cette heuristique.

Cependant, l'algorithme du peintre sur un arbre construit
avec l'heuristique $H_1$ est souvent plus rapide ou dans le
pire des cas aussi rapide que les deux autres approches.

\subsubsection*{Fichiers \og\emph{random}\fg}
Les fichiers de type \og\emph{random}\fg{} sont constitués de polygones
aléatoires disposés dans le plan partitionné en petits carrés, contenant
chacun un polygone. Les tables \ref{tab:rand1}, \ref{tab:rand2},
\ref{tab:rand3} et \ref{tab:rand4} reprennent
les valeurs des tests effectués avec ce type de fichiers.

Sur ces fichiers, l'heuristique $H_1$
possède la plus petite hauteur et la plus petite taille.
Sur ces mêmes fichiers, l'heuristique aléatoire a une même hauteur moyenne
que l'heuristique $H_1$ mais une taille plus grande.
\subsubsection*{Fichiers \og\emph{ellipses}\fg}
Les fichiers \og\emph{ellipses}\fg{} sont formés de segments disposés de manière
à délimiter des ellipses concentriques. Très peu de segments
sont colinéaires sur ce type de fichier.
Les tables \ref{tab:ell1}, \ref{tab:ell2} et \ref{tab:ell3} reprennent les
résultats des expériences effectuées.

Sur ces fichiers, on remarque que la taille des arbres
est relativement identique pour chaque heuristique.
Toutefois, la hauteur varie; les tests aléatoires présentent
une hauteur moindre que ceux de l'heuristique $H_1$, dont les arbres
sont moins hauts que ceux produits par l'heuristique linéaire.

\subsubsection*{Fichiers \og\emph{rectangles}\fg}
Les fichiers \og\emph{rectangles}\fg{} sont constitués de segments
délimitant des rectangles emboîtés. Ils contrastent avec les fichiers
de type \og\emph{ellipses}\fg{} car ils contiennent beaucoup de segments
colinéaires. Les tables \ref{tab:rect1}, \ref{tab:rect2}, \ref{tab:rect3}
et \ref{tab:rect4} contiennent
l'ensemble des résultats expérimentaux.

Pour ces fichiers, l'heuristique aléatoire possède
la plus petite hauteur mais aussi la plus grande taille.
De plus, hormis pour le fichier \texttt{rectanglesHuge.txt}, $H_1$
et l'heuristique linéaire sont similaires.

Le fichier \texttt{rectanglesHuge.txt} possède la particularité d'avoir des
rectangles qui ne sont pas contenus l'un dans l'autre, ce qui explique
les performances différentes relevées pour ce fichier-là par rapport aux
autres du même type.

%%%Local Variables:
%%% mode: latex
%%% TeX-master: "../rapportGp1"
%%% End:
