\subsection{Analyse des temps CPU}
Les tests ont été effectués avec un \oe{}il à la position $(0,0)$ et un
angle de 1 radian. Les données employées pour les tests sont les scènes
qui nous ont été fournies sur la plateforme e-learning.

La configuration du PC employé pour les tests était la suivante :

\begin{center}
  Intel Core i5-7300HQ CPU @ 2.50GHz $\times$ 4,  8Gb ram, système Ubuntu 18.04.2 LTS,
  Java version "1.8.0\_191",
\end{center}
Pour calculer le temps CPU, la méthode \texttt{nanotime} de la classe
\texttt{System} a été utilisée.

Un sous-ensemble des valeurs expérimentales est repris au sein de
ce document, les autres sont consignées dans un document complémentaire
à celui-ci.

Par la suite, la formulation la hauteur de l'heuristique $X$ sera abusivement
employée pour parler de la hauteur de l'arbre générée par l'heuristique $X$
(sous-entendu hauteur moyenne dans le cas de l'heuristique aléatoire).
Il en va de même pour les autres caractéristiques.

\subsubsection*{Phénomènes globaux}
Globalement, on peut remarquer que l'heuristique $H_1$ prend beaucoup plus de
temps à construire l'arbre que les autres heuristiques.
Comme attendu par la proposition \ref{compl:painter}, on remarque
que si une heuristique a une taille moins importante qu'une autre
alors l'algorithme du peintre est plus efficace pour cette heuristique.

Cependant, l'algorithme du peintre sur un arbre construit
avec l'heuristique $H_1$ est souvent plus rapide ou dans le
pire des cas aussi rapide que les deux autres approches.

\subsubsection*{Fichiers \og\emph{random}\fg}
Les fichiers de type \og\emph{random}\fg{} sont constitués de polygônes
aléatoires disposés dans le plan partitionné en petits carrés, contenant
chacun un polygône. Les tables \ref{tab:rand1} et \ref{tab:rand2} reprennent
certaines valeurs des tests effectués avec ce type de fichiers.

\begin{table}
\caption{randomMedium.txt}\label{tab:rand1}
\begin{center}
{\tabulinesep=1.2mm
\begin{tabu}{|l|c|c|c|}
  \hline
  & Heuristique linéaire  & Heuristique 1 & Heuristique aléatoire \\ 
  \hline
  Taille &     34153&     25559&   27707,5  \\ 
  \hline
  Hauteur &        96&        35&      38,0  \\ 
  \hline
  Temps constructeur (ms) &          8,5829&       1255,3063&         12,1229  \\ 
  \hline
  Temps peintre (ms) &           28,6280&          21,2297&          23,0759  \\ 
  \hline
\end{tabu}
}
\end{center}

%%%Local Variables:
%%% mode: latex
%%% TeX-master: "../rapportGp1"
%%% End:
\end{table}

\begin{table}
\caption{randomHuge.txt}\label{tab:rand2}
\begin{center}
{\tabulinesep=1.2mm
\begin{tabu}{|l|c|c|c|}
  \hline
  & Heuristique linéaire  & Heuristique 1 & Heuristique aléatoire \\ 
  \hline
  Taille &    141415&    105197&  111894,9  \\ 
  \hline
  Hauteur &       157&        50&      49,4  \\ 
  \hline
  Temps constructeur (ms) &         53,4573&      25755,7597&         51,0537  \\ 
  \hline
  Temps peintre (ms) &          124,8024&          89,8473&          94,6599  \\ 
  \hline
\end{tabu}
}
\end{center}

%%%Local Variables:
%%% mode: latex
%%% TeX-master: "../rapportGp1"
%%% End:
\end{table}

Sur ces fichiers, l'heuristique $H_1$
possède la plus petite hauteur et la plus petite taille.
Sur ces mêmes fichiers, l'heuristique aléatoire a une même hauteur moyenne
que l'heuristique $H_1$ mais une taille plus grande.
\subsubsection*{Fichiers \og\emph{ellipses}\fg}
Les fichiers \og\emph{ellipses}\fg{} sont formés de segments disposés de manière
à délimiter des ellipses concentriques. Très peu de segments
sont colinéaires sur ce type de fichier.
Les tables \ref{tab:ell1} et \ref{tab:ell2} reprennent quelques
résultats des expériences effectuées.

\begin{table}
\caption{ellipsesSmall.txt}\label{tab:ell1}
\begin{center}
{\tabulinesep=1.2mm
\begin{tabu}{|l|c|c|c|}
  \hline
  & Heuristique linéaire  & Heuristique 1 & Heuristique aléatoire \\ 
  \hline
  Taille &       401&       401&     421,7  \\ 
  \hline
  Hauteur &       201&       155&      59,6  \\ 
  \hline
  Temps constructeur (ms) &          0,2428&          7,1014&          0,0603  \\ 
  \hline
  Temps peintre (ms) &            0,3253&           0,3205&           0,3525  \\ 
  \hline
\end{tabu}
}
\end{center}

%%%Local Variables:
%%% mode: latex
%%% TeX-master: "../rapportGp1"
%%% End:
\end{table}

\begin{table}
\caption{ellipsesLarge.txt}\label{tab:ell2}
\begin{center}
{\tabulinesep=1.2mm
\begin{tabu}{|l|c|c|c|}
  \hline
  & Heuristique linéaire  & Heuristique 1 & Heuristique aléatoire \\ 
  \hline
  Taille &      9001&      9087&    9116,7  \\ 
  \hline
  Hauteur &      4501&      1713&     516,8  \\ 
  \hline
  Temps constructeur (ms) &        114,2485&      15021,3817&          3,4645  \\ 
  \hline
  Temps peintre (ms) &            7,6303&           7,7376&           8,0163  \\ 
  \hline
\end{tabu}
}
\end{center}

%%%Local Variables:
%%% mode: latex
%%% TeX-master: "../rapportGp1"
%%% End:
\end{table}

Sur ces fichiers, on remarque que la taille des arbres
est relativement identique pour chaque heuristique.
Toutefois, la hauteur quant à elle varie; les tests aléatoires présentent
une hauteur moindre que ceux de l'heuristique $H_1$, dont les arbres
sont moins haut que ceux produits par l'heuristique linéaire.

\subsubsection*{Fichiers \og\emph{rectangles}\fg}
Les fichiers \og\emph{rectangles}\fg{} sont constitués de segments
délimitant des rectangles emboîtés. Ils contrastent avec les fichiers
de type \og\emph{ellipses}\fg{} car ils contiennent beaucoup de segments
colinéaires. Les tables \ref{tab:rect1} et \ref{tab:rect2} contiennent
un sous-ensemble des résultats expérimentaux.

\begin{table}
\caption{rectanglesLarge.txt}\label{tab:rect1}
\begin{center}
{\tabulinesep=1.2mm
\begin{tabu}{|l|c|c|c|}
  \hline
  & Heuristique linéaire  & Heuristique 1 & Heuristique aléatoire \\ 
  \hline
  Taille &        89&        89&     334,5  \\ 
  \hline
  Hauteur &        45&        45&      15,1  \\ 
  \hline
  Temps constructeur (ms) &          1,7089&       2164,4079&          1,0345  \\ 
  \hline
  Temps peintre (ms) &           11,5583&          11,5322&          11,6792  \\ 
  \hline
\end{tabu}
}
\end{center}

%%%Local Variables:
%%% mode: latex
%%% TeX-master: "../rapportGp1"
%%% End:
\end{table}

Pour ces fichiers, l'heuristique aléatoire possède
la plus petite hauteur mais aussi la plus grande taille.
De plus, hormis pour le fichier \texttt{rectanglesHuge.txt}, $H_1$
et l'heuristique linéaire sont similaires.

\begin{table}
\caption{rectanglesHuge.txt}\label{tab:rect2}
\begin{center}
{\tabulinesep=1.2mm
\begin{tabu}{|l|c|c|c|}
  \hline
  & Heuristique linéaire  & Heuristique 1 & Heuristique aléatoire \\ 
  \hline
  Taille &       115&       311&     313,5  \\ 
  \hline
  Hauteur &        46&        24&      15,0  \\ 
  \hline
  Temps constructeur (ms) &          5,2832&       3033,3284&          2,2017  \\ 
  \hline
  Temps peintre (ms) &           27,9096&          29,7665&          26,5652  \\ 
  \hline
\end{tabu}
}
\end{center}

%%%Local Variables:
%%% mode: latex
%%% TeX-master: "../rapportGp1"
%%% End:
\end{table}

Le fichier \texttt{rectanglesHuge.txt} possède la particularité d'avoir des
rectangles qui ne sont pas contenus l'un dans l'autre, ce qui explique
les performances différentes relevées pour ce fichier là par rapport aux
autre du même type.

%%%Local Variables:
%%% mode: latex
%%% TeX-master: "../rapportGp1"
%%% End:
