Analyse des temps cpu.
les tests ont été effectués avec un oeil à la position 0;0 et un angle de 1.
la configuration du pc était la suivante :

\begin{center}
  Intel Core i5-7300HQ CPU @ 2.50GHz × 4,  8Gb ram, système Ubuntu 18.04.2 LTS, java version "1.8.0_191",
  Pour calculer le temps cpu, la méthode nanotime de la classe System a été utilisée. 
\end{center}

Par la suite, on utilisera abusivement la hauteur de l'heuristique "X" pour la hauteur de l'arbre générée par l'heuristique "X". Il en va de même pour les autres caractèristiques.


On peut remarquer que l'heuristique 1 prend beaucoup plus de temps à construire l'arbre que les autres heuristiques.
Comme attendu par la proposition \ref{compl:painter}, on remarque que si une heuristique a une taille moins importante qu'une autre alors l'algoritme du peintre sera plus efficace pour cette heuristique.

Cependant, l'algorithme du peintre sur un arbre construit avec cette heuristique est souvent plus rapide ou dans le pire des cas aussi rapide.

De plus, sur les fichiers du type "random", l'heuristique 1 possède la plus petite hauteur et la plus petite taille.
ur ces mêmes fichiers, l'heuristique aléatoire  a une même hauteur que l'heuristique 1 mais une taille plus grande. 

Les fichiers du dossier "first" sont trop petits pour voir une différence significative. Les 3 heuristiques sont assez équivalentes sauf l'heuristique aléatoire qui a une plus grande taille et l'heuristique linéaire une plus grande hauteur.

Sur les fichiers "ellipses", on remarque que la taille des arbres est relativement identiques mais la hauteur quant à elle suit l'ordre suivant :

\begin{center}
  aléatoire < première < linéaire
\end{center}

Enfin, les fichiers "rectangles"  



%%%Local Variables:
%%% mode: latex
%%% TeX-master: "../rapportGp1"
%%% End:
