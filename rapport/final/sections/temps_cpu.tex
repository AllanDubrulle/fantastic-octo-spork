\subsection{Analyse des temps CPU}
Les tests ont été effectués avec un \oe{}il à la position $(0,0)$ et un
angle de 1 radian.
La configuration du pc était la suivante :

\begin{center}
  Intel Core i5-7300HQ CPU @ 2.50GHz $\times$ 4,  8Gb ram, système Ubuntu 18.04.2 LTS,
  Java version "1.8.0\_191",
\end{center}
Pour calculer le temps cpu, la méthode \texttt{nanotime} de la classe
\texttt{System} a été utilisée.

Par la suite, la formulation la hauteur de l'heuristique $X$ sera abusivement
employée pour parler de la hauteur de l'arbre générée par l'heuristique $X$.
Il en va de même pour les autres caractéristiques.

On peut remarquer que l'heuristique $H_1$ prend beaucoup plus de
temps à construire l'arbre que les autres heuristiques.
Comme attendu par la proposition \ref{compl:painter}, on remarque
que si une heuristique a une taille moins importante qu'une autre
alors l'algorithme du peintre est plus efficace pour cette heuristique.

Cependant, l'algorithme du peintre sur un arbre construit
avec cette heuristique est souvent plus rapide ou dans le
pire des cas aussi rapide.

De plus, sur les fichiers du type "random", l'heuristique $H_1$
possède la plus petite hauteur et la plus petite taille.
Sur ces mêmes fichiers, l'heuristique aléatoire  a une même hauteur moyenne
que l'heuristique 1 mais une taille plus grande.

Sur les fichiers "ellipses", on remarque que la taille des arbres
est relativement identique mais la hauteur quant à elle suit l'ordre suivant :

\begin{center}
  aléatoire < $H_1$ < linéaire
\end{center}

Enfin, les fichiers "rectangles"



%%%Local Variables:
%%% mode: latex
%%% TeX-master: "../rapportGp1"
%%% End:
