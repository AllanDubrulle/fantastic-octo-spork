\subsection{Heuristique H1}
Soient $S$ un ensemble de segment, $s$ et $s'$ deux segments
de $S$ tels que $s \ne s'$.
Plusieurs configurations sont possibles.
\begin{enumerate}
\item les droites  $D_s$ et  $D_{s'}$ sont confondues. \label{h1:enum1}
\item la droite $D_s$ parallèle à $D_{s'} $. \label{h1:enum2}
\item les segments $s$ et $s'$ ont une extrémité en commun \label{h1:enum3}
\item la droite $D_{s'}$ intersecte $s$ en un point n'appartenant pas à $s'$.\label{h1:enum4}
\item le symétrique du cas précédent. \label{h1:enum5}
\end{enumerate}

Il est important de remarquer que ces cas sont deux-à-deux disjoints
et donc en particulier \ref{h1:enum4} et \ref{h1:enum5} sont disjoints.
Supposons que pour tout $s$, $s' \in S $ tel que $s \neq s'$,
les segments $s$ et $s'$ n'ont pas une extrémité en commun.

Soit $g_s =$ nombre de droites $D'$ supportant un segment $s' \neq s$ de $S$ qui intersectent
le segments.

Maximiser $g_s$ pour $s \in S$ revient à choisir le segment $s$ qui maximise
le nombre de cas \ref{h1:enum4} pour $s' \in S$.
Par ce choix, le nombre de cas \ref{h1:enum5} est minimisé.
Par conséquent, la droite $D_s$ intersecte un nombre minimal de segment $s'$.
Ce qui équivaut au fait que $f_D$ sera minimal.


Remarquons que pour l'heuristique H1, les 2 premiers cas ne sont pas exploités.
Tandis que le cas \ref{h1:enum3} est un cas pathologique n'étant pas résolu par H1.
L'efficacité de H1 est donc limitée si le nombre de cas
\ref{h1:enum3} est important.

H1 réduit le nombre de fragments pour réduire la taille de l'arbre
et donc améliorer le temps de l'algorithme du peintre qui dépend
fortement de la taille de l'arbre et du nombre de fragments générés (par
le corollaire \ref{peintre:cor}).

\begin{algorithm}
  \caption{H1($S$)}
  \begin{algorithmic}[1] \label{algo:propvis}
    \REQUIRE $S$ un ensemble de segment.
    \ENSURE $s$ tel que $g_s$ est maximal.
    \FOR{$s \in S$}
    \FOR{$s' \in S$}
    \IF{$s \neq s'$ et  $D_{s'}$ intersecte $s$}
    \STATE $g_s++$
    \ENDIF
    \ENDFOR
    \ENDFOR
    \RETURN $max\{g_s|s\in S\}$
  \end{algorithmic}
\end{algorithm}

Soit $N = Card(S)$.
\begin{prop}
  L'algorithme H1 est en $O(N^2)$.
\end{prop}
\begin{proof}
  L'algorithme itère sur chaque segment $s \in S$.
  On réitère ensuite sur chaque segment $s' \in S$.
  Tandis que l'intérieur des boucles correspond à du calcul et
  est donc en $O(1)$.  On obtient que H1 est en $N*(N*O(1)) = O(N^2)$
\end{proof}



%%% Local Variables:
%%% mode: latex
%%% TeX-master: "../rapportGp1"
%%% End:
