\subsection{Heuristiques H1}
\subsubsection{Raisonnement}
Soientt $v$ un noeud de l'arbre. $s$ et $s'$ deux segments de $I_v$ tel que $s \ne s'$.

Plusieurs configurations sont possibles.
\begin{enumerate}
\item  $D_s \eq D_{s'} $.
\item $D_s$ parrallèle à $D_{s'} $.
\item $s$ et $s'$ s'intersecte
\item $D_{s'}$ intersecte $s$ en un point n'appartenant pas à $s'$.
\item le symétrique du cas précédent.
\end{enumerate}

Il est important de remarquer que ces cas sont 2 à 2 disjoints et donc en particulier 4 et 5 sont disjoints.
Supposons que pour tout $s$,$s' \in I_v $ tel que $s \neq s'$, $s$ n'intersecte pas $s'$
Le principe de H1 est de choisir le segment $s$ qui maximise le nombre de cas 4 pour $s' \in I_v$.
Par ce choix, le nombre de cas 5 est minimisé.
Par conséquent, la droite $D_s$ intersecte un nombre minimal de segment $s'$.
Ce qui équivaut au fait que $f_D$ sera minimal.


Remarquons que pour l'heuristique H1, les 2 premiers cas ne sont pas exploités.
Tandis que le cas 3 est un cas pathologique ne pouvant pas être résolu par H1.
l'efficacité de H1 est donc limité si le nombre de cas 3 est important.

H1 réduit le nombre de fragments pour réduire la taille de l'arbre et donc améliorer le temps de l'algorithme du peintre qui dépend fortement de la taille de l'arbre.

\subsubsection{Algorithme}
\begin{algorithm}
  \caption{H1($I_v$)}
  \begin{algorithmic}[1] \label{algo:propvis}
    \REQUIRE $I_v$ l'ensemble des segments du noeud $v$.
    \ENSURE $s$ tel que $g_s$ est maximal.
    \FOR{$s \in I_v$}
    \FOR{$s' \in I_v$}
    \IF{$s \neq s'$ et  $D_{s'}$ intersecte $s$}
    \state $g_s++$
    \ENDIF
    \ENDFOR
    \ENDFOR
    \RETURN $max\{g_s|s\in I_v\}$
  \end{algorithmic}
\end{algorithm}
\subsubsection{Complexité}
Soient $v$ un noeud de l'arbre. L'algorithme itère sur chaque segment $s \in I_v$.
On réitère ensuite sur chaque segment $s' \in I_v$. Tandis que l'intérieur des boucles correspond à du calcul et est donc en $O(1)$.
Soit $N \eq Card(I_v)$. On obtient que H1 est en $O(N²)$
