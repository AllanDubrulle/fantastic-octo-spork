\subsection{Heuristique H1}
\subsubsection*{Raisonnement}
Soient $v$ un noeud de l'arbre, $s$ et $s'$ deux segments
de $I_v$ tels que $s \ne s'$.
% rappeler ce que I_v, je sais pas si tu comptais
% le faire avant cette sous-sous-section
Plusieurs configurations sont possibles.
\begin{enumerate}
\item  $D_s = D_{s'} $. \label{h1:enum1}
\item $D_s$ parallèle à $D_{s'} $. \label{h1:enum2}
\item $s$ et $s'$ s'intersectent \label{h1:enum3}
\item $D_{s'}$ intersecte $s$ en un point n'appartenant pas à $s'$.\label{h1:enum4}
\item le symétrique du cas précédent. \label{h1:enum5}
\end{enumerate}

Il est important de remarquer que ces cas sont deux-à-deux disjoints
et donc en particulier \ref{h1:enum4} et \ref{h1:enum5} sont disjoints.
Supposons que pour tout $s$, $s' \in I_v $ tel que $s \neq s'$,
$s$ n'intersecte pas $s'$

Le principe de H1 est de choisir le segment $s$ qui maximise
le nombre de cas \ref{h1:enum4} pour $s' \in I_v$.
Par ce choix, le nombre de cas \ref{h1:enum5} est minimisé.
Par conséquent, la droite $D_s$ intersecte un nombre minimal de segment $s'$.
Ce qui équivaut au fait que $f_D$ sera minimal.


Remarquons que pour l'heuristique H1, les 2 premiers cas ne sont pas exploités.
Tandis que le cas \ref{h1:enum3} est un cas pathologique ne pouvant pas
être résolu par H1.
L'efficacité de H1 est donc limitée si le nombre de cas
\ref{h1:enum3} est important.

H1 réduit le nombre de fragments pour réduire la taille de l'arbre
et donc améliorer le temps de l'algorithme du peintre qui dépend
fortement de la taille de l'arbre et du nombre de fragments générés (par
le corollaire \ref{peintre:cor}).

\subsubsection*{Algorithme}
\begin{algorithm}
  \caption{H1($I_v$)}
  \begin{algorithmic}[1] \label{algo:propvis}
    \REQUIRE $I_v$ l'ensemble des segments du noeud $v$.
    \ENSURE $s$ tel que $g_s$ est maximal.
    \FOR{$s \in I_v$}
    \FOR{$s' \in I_v$}
    \IF{$s \neq s'$ et  $D_{s'}$ intersecte $s$}
    \STATE $g_s++$
    \ENDIF
    \ENDFOR
    \ENDFOR
    \RETURN $max\{g_s|s\in I_v\}$
  \end{algorithmic}
\end{algorithm}
\subsubsection*{Complexité}
Soient $v$ un noeud de l'arbre. L'algorithme itère sur chaque segment
$s \in I_v$.
On réitère ensuite sur chaque segment $s' \in I_v$.
Tandis que l'intérieur des boucles correspond à du calcul et
est donc en $O(1)$. Soit $N = Card(I_v)$. On obtient que H1 est en $O(N^2)$


%%% Local Variables:
%%% mode: latex
%%% TeX-master: "../rapportGp1"
%%% End:
