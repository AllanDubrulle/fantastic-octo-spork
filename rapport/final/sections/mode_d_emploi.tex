\subsection{Compilation}
% Todo
\subsection{Programme en console}
le programme en console s'exécute avec la commande :

java -jar target/BSPPainter-TestConsole.jar nomdufichier.txt

Si le fichier n'est pas trouvable ou n'existe pas, le programme s'arrêtera après avoir déclenché une IOException.

De plus, Le fichier passé en argument doit être au format spécifique des scènes. Dans le cas contraire, le programme lancera une FileFormatException et se coupera.

Le programme effectue une partie des calculs au démarrage. Quelques secondes peuvent donc s'écoulaient avant le moindre affichage si le fichier est conséquent.

Le premier panneau d'affichage demande d'entrer un nombre entre 1 et 5. Pour chaque valeur, le programme spécifie les informations affichés.

La console bouclera sur ce panneau tant que l'entrée de l'utilisateur n'est pas un nombre ou n'est pas compris entre 1 et 5.

Si l'utilisateur choisit d'afficher les performances de l'algorithme du peintre, un nouveau menu s'affichera demandant à l'utilisateur d'entrer les positions de l'oeil. Le format est précisé dans le menu et est le suivant :

En premier, la coordonnée x de l'oeil en nombre flottant suivie du caractère ';'.

Deuxième, la coordonnée y de l'oeil en nombre flottant suivie du caractère';'.

Finalement, l'angle qui correspond à la direction de l'oeil tel que décrit dans la section \ref{not:oeil}.
% faire le lien vers la section 2.3.1

Si l'entrée contient trop de caractère séparé par ';', l'entrée sera rejettée. Il en va de même si l'une des 3 valeurs n'est pas un nombre. 

Le programme affichera alors les informations souhaitées et lancera une nouvelle boucle similaire à la première. Cette fois-ci, Les nombres à entrer sont 1 ou 0. 0 ferme le programme tandis que 1 remet le programme dans la précédente boucle.

%%%Local Variables:
%%% mode: latex
%%% TeX-master: "../rapportGp1"
%%% End:
