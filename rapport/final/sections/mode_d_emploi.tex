\subsection{Compilation}
La compilation du programme s'effectue en ligne de commande à l'aide
de l'outil Maven (\url{https://maven.apache.org/}). Dans la racine
du projet (où se trouve le fichier \texttt{pom.xml}),
via la ligne de commande, l'instruction
\begin{center}
  \texttt{
    mvn install
  }
\end{center}
compile le projet et les tests unitaires, exécute les tests unitaires
et génère trois fichiers au format \emph{.jar} dans le répertoire
\texttt{target}:
\begin{itemize}
\item le programme avec interface graphique correspond au fichier
  \\ \texttt{BSPPainter-TestGraphique-1.0.jar};
\item le programme en ligne de commande correspond au fichier
  \\ \texttt{BSPPainter-TestConsole-jar-with-dependencies.jar};
\item la javadoc du projet est générée dans l'archive (non exécutable)
  \\ \texttt{BSPPainter-TestGraphique-1.0-javadoc.jar}.
\end{itemize}

\subsection{Programme en console}
Le programme en console s'exécute avec la commande :
\begin{center}
  \texttt{
    java -jar BSPPainter-TestConsole-jar-with-dependencies.jar nomdufichier
  }
\end{center}

Si le fichier n'est pas trouvable ou n'existe pas, le programme
s'arrêtera après avoir déclenché une \texttt{IOException}.
Le fichier soumis doit être au format spécifique des scènes.
Dans le cas contraire, le programme lancera une \texttt{FileFormatException}
et se coupera.

Le programme effectue une partie des calculs au démarrage.
Quelques secondes peuvent donc s'écouler avant le moindre affichage
si le fichier est conséquent.

Le premier panneau d'affichage demande d'entrer un
nombre entre $1$ et $5$. Pour chaque valeur, le programme
spécifie les informations affichées.
La console bouclera sur ce panneau tant que
l'entrée de l'utilisateur n'est pas un nombre
compris entre $1$ et $5$.

Si l'utilisateur choisit d'afficher les performances de l'algorithme
du peintre, un nouveau menu s'affichera demandant à l'utilisateur
d'entrer les positions de l'oeil. Le format est précisé
dans le menu et est le suivant :
\begin{itemize}
\item En premier, la coordonnée $x$ de l'oeil en nombre
  flottant suivie du caractère \og;\fg{}.
\item Deuxièmement, la coordonnée $y$ de l'oeil en
  nombre flottant suivie du caractère \og;\fg.
\item Finalement, l'angle qui correspond à la direction
  de l'oeil tel que décrit dans la section \ref{not:oeil}.
\end{itemize}
Un exemple d'entrée valide est:
\begin{center}
  \texttt{
    100.45;235.5;2.57
  }
\end{center}

Si l'entrée contient trop de caractères séparés par des points-virgules,
l'entrée sera rejetée. Il en va de même si l'une des 3 valeurs fournies
n'est pas un nombre.
Le programme affichera alors les informations souhaitées.

Après cet affichage, le programme propose deux options numérotées par
$0$ et $1$: $0$ ferme le programme tandis que $1$ affiche à nouveau
le premier menu.

%%%Local Variables:
%%% mode: latex
%%% TeX-master: "../rapportGp1"
%%% End:
