%Document contenant l'introduction de l'exerice
%préliminaire du projet de structures de données II

Nous nous intéressons ici au problème suivant: étant donnés
un arbre BSP $T$ représentant une scène dans le plan $\mathbb{R}^2$
et deux points $x$ et $y$ du plan, déterminer si le segment $[x, y]$
est dans l'arbre $T$.

Posons les notations qui seront utilisées tout au long de la
résolution de l'exercice. On dénote par $T$ sans distinction la
racine et l'arbre correspondant.

Si $T$ est une feuille, on dénote par $S_T$ l'ensemble des
(fragments de) segments contenus dans cette feuille.
Notez que $S_T$ contient au plus un élément (par définition de BSP).

Si $T$ est un noeud interne, on note $D$ l'hyperplan affin stocké
en ce noeud. Ce dernier est donné par une équation $f_D(x_1, x_2) = 0$
où $f_D: \mathbb{R}^2 \to \mathbb{R}: (x_1, x_2)\mapsto a x_1 + b x_2 +c$
pour certains $a, b, c\in\mathbb{R}$. Ceci sépare le plan en deux
parties: $$D^+=\{(x_1, x_2)\in\mathbb{R}^2\mid f_D(x_1, x_2) > 0\}$$ et
$$D^-=\{(x_1, x_2)\in\mathbb{R}^2\mid f_D(x_1, x_2) < 0\}$$
Ceci motive la notation $T^-$ (resp. $T^+$) pour le fils gauche
(resp. droit) de $T$ représentant les fragments contenus dans $D^-$
(resp. $D^+$). L'ensemble des segments contenus dans $D$ est noté
$S_T$ de la même manière que pour les feuilles.
