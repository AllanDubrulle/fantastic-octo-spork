% Document pour donner l'algorithme de recherche
% de segment dans un BSP donné.
% A input dans le fichier principal.

\begin{algorithm}
  \caption{Recherche\_segment($T, x, y$)}
  \begin{algorithmic}[1] \label{algo:main}
    \REQUIRE Un arbre BSP $T$, deux points $x$ et $y$ du plan réel.
    \ENSURE Vrai si et seulement si le segment $[x, y]$ est contenu dans $T$.
    \IF{$T$ est réduit à une feuille}
    \RETURN $[x, y]\in S_T$
    \ENDIF
    \IF{$f_D(x) > 0$}
      \IF{$f_D(y) \geq 0$}
      \RETURN Recherche\_segment($T^+$, $x$, $y$)
      \ELSE
      \STATE $z \leftarrow D \cap [x, y]$
      \RETURN Recherche\_segment($T^+$, $x$, $z$) $\land$
              Recherche\_segment($T^-$, $z$, $y$)
      \ENDIF
      \ELSE
      \IF{$f_D(x) = f_D(y) = 0$}
      \RETURN $[x, y]\in S_T$
      \ELSIF{$f_D(y) \leq 0$}
      \RETURN Recherche\_segment($T^-$, $x$, $y$)
      \ELSE
      \STATE $z \leftarrow D \cap [x, y]$
      \RETURN Recherche\_segment($T^-$, $x$, $z$) $\land$
              Recherche\_segment($T^+$, $z$, $y$)
      \ENDIF
    \ENDIF
  \end{algorithmic}
\end{algorithm}

L'algorithme vérifiant la présence d'un segment donné est détaillé à
l'algorithme \ref{algo:main}.

Les détails suivants sont omis afin de ne pas alourdir l'algorithme:
l'appartenance dans les ensembles stockés en $T$ (ensembles notés
$S_T$) et le calcul de l'intersection du segment et de la
droite stockée en $T$.

Afin de vérifier l'appartenance d'un segment à un ensemble, il
suffit de parcourir l'ensemble, représenté ici par une liste
chaînée, et de tester l'égalité entre les éléments et le segment
en entrée.

Il suffit donc d'expliciter l'égalité de segments et d'appliquer
un algorithme de recherche classique dans une liste chaînée.
Deux segments sont égaux si et seulement s'ils ont les
mêmes extrémités, c'est-à-dire:
$$ [x_1, y_1] = [x_2, y_2] \iff
(x_1 = x_2 \land y_1 = y_2) \lor (x_1 = y_2 \land y_1 = x_2) $$

Il reste à discuter le calcul de l'intersection d'un segment
et d'une droite dans le cas où elle n'est pas vide. Or, ces
considérations sont détaillées dans le raisonnement mathématique
(se référer à l'égalité (\ref{eq:t}) et se rappeler que $z = tx + (1-t) y$).
