% Explication du raisonnement mathématique derrière
% l'algorithme, donné dans une section différente de
% ce dernier

\subsection{Cas de base}
La question présente deux cas de base exclusifs: l'arbre $T$ est
réduit à une feuille ou les points $x, y$ délimitant le segment
recherché sont contenus dans l'hyperplan stocké en la racine de $T$.

Si $T$ est réduit à une feuille, il suffit de tester si le segment
stocké en cette feuille (s'il existe) correspond à $[x, y]$. Si ce
n'est pas le cas ou que la feuille est vide, alors l'algorithme
retourne faux.

Si les points $x$ et $y$ sont contenus dans $D$, alors par convexité
de $D$, le segment est contenu dans $D$. Le problème se réduit alors
à la recherche d'une donnée au sein d'une liste chaînée.

\subsection{Cas général}
Supposons que $T$ possède deux fils (ie. $T$ n'est pas une feuille).
Discutons les différents cas.
