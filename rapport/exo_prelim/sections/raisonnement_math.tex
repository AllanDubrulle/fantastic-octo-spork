% Explication du raisonnement mathématique derrière
% l'algorithme, donné dans une section différente de
% ce dernier

\subsection{Cas de base}
La question présente deux cas de base exclusifs: l'arbre $T$ est
réduit à une feuille ou les points $x, y$ délimitant le segment
recherché sont contenus dans l'hyperplan stocké en la racine de $T$.

Si $T$ est réduit à une feuille, il suffit de tester si le segment
stocké en cette feuille (s'il existe) correspond à $[x, y]$. Si ce
n'est pas le cas ou que la feuille est vide, alors l'algorithme
retourne faux.

Si les points $x$ et $y$ sont contenus dans $D$, alors par convexité
de $D$, le segment est contenu dans $D$. Le problème se réduit alors
à la recherche d'une donnée au sein d'une liste chaînée.

\subsection{Cas général}
Supposons que $T$ possède deux fils (ie. $T$ n'est pas une feuille).
Discutons les différents cas.

Si $x,y \in D^+$ (resp. $D^-$), alors le segment $[x,y] \subseteq D^+$
(resp. $D^-$) par convexité. Par définition d'arbre BSP,
le segment recherché est contenu dans $T$. si et seulement s'il
est dans $T^+$ (resp. $T^-$). Il suffit donc de poursuivre la
recherche récursivement dans $T^+ $ (resp. $T^-$).

Si $x \in D $ et $y \in D^+$ (resp. $D^-$), alors le segment $[x, y]$ est
contenu dans le demi-plan $D^+$ (resp. $D^-$).Il suffit donc de poursuivre la
recherche récursivement comme précédemment. Le cas $y \in D $ et $x \in D^+$
(resp. $D^-$) est symétrique au cas précédent.

Il reste deux cas à considérer: le premier est $ x \in D^+$ et $ y \in D^-$ et
le second est $x \in D^-$ et $ y \in D^+$. Les deux cas étant symétriques on
supposera sans perte de généralité que $ x \in D^+$ et $ y \in D^-$.
Le segment $[x,y]$ intersecte $D$ en un point qu'on nommera $z$.

Savoir si le segment $[x,y]$ se trouve dans $T$ revient donc à savoir si
les segments $[x,z]$ et $[z,y]$ (le segment est fragmenté) sont
éléments de $T$ et par conséquent de faire deux recherches
récursivement (cfr. cas précédent) et de retourner vrai si et
seulement si les deux appels retournent vrai.

Cependant, il faut connaître $z$ pour appliquer ce raisonnement. Nous allons
donc montrer comment déterminer $z$ à partir de $x$, $y$ et $f_D$.
Posons $g: \mathbb{R}^2 \to \mathbb{R}$ la fonction définie par
$g (x_1, x_2) =  a x_1 + b x_2$ pour tout $(x_1, x_2)\in\mathbb{R}^2$.
Nous avons que $f_D = g + c $ et la fonction $g$ est linéaire.

De plus, nous avons que $f_D(z) = 0$ et $\exists t \in [0,1]$ tel que
$ z = tx+ (1-t)y$
\footnote{En effet, on a par définition: $[x, y] = \{tx + (1 - t)y\mid t\in [0,1]\}$}.
Donc, par linéarité de $g$:
\begin{equation}
  0 = f_D(z)  = f_D (tx+ (1-t)y) = t(g(x)-g(y))+g(y)+c
\end{equation}

Par conséquent,
\begin{equation} \label{eq:t}
    t = \frac{(-c - g(y))}{(g(x)-g(y))} = \frac{-(f_D(y)+2c)}{f_D(x)-f_D(y)}
\end{equation}
L'expression de $t$ est bien définie car  $g(x) \neq g(y)$ car $ g(y) < -c  $ et $g(x) > -c$.
