\subsection{Coûts locaux}
Etudions le coût local dans une feuille.
Par définition, $S_T$ contient au plus une donnée.
Donc, le coût local par feuille est en temps constant $O(1)$.

Intéressons-nous désormais au coût local par noeud interne.
Dans un premier temps, si le segment est contenu dans
la droite stockée dans le noeud, alors on a un coût linéaire
en le nombre de fragments dans $S_T$ (borné par le nombre de
segments employés dans la construction de l'arbre).

Sinon, seules des opérations élémentaires sont
effectuées (opérations arithmétiques, appels de fonctions
et comparaisons), ce qui entraîne un coût local en $O(1)$.

\subsection{Nombre d'appels récursifs (pire des cas)}
Remarquons d'abord que dans l'algorithme, chaque noeud est
visité au plus une fois.
Par conséquent, le nombre de noeuds visités maximal correspond
au nombre de noeuds de $T$.
Il est possible que tous les noeuds de l'arbre soient visités;
si à chaque noeud interne traité, (le fragment considéré du)
segment recherché intersecte la droite correspondant au noeud,
cela entraîne un appel récursif sur chacun des fils du noeud.
Le nombre d'appels récursifs est en $O(N)$ où $N$ correspond
au nombre de noeuds de l'arbre passé en paramètre.

Par conséquent, on en déduit une majoration grossière de
la complexité en $O(N \times n)$ où $n$ est le nombre de
segments stockés dans $T$, étant donné que chaque liste contient
au plus $n$ fragments.

\subsection{Nombre d'appels récursifs (cas simple)}
Supposons que $[x,y]$ n'intersecte aucune droite contenue dans $T$.
Alors l'algorithme suit un chemin de l'arbre. Nous avons donc
au maximum $h$ appels récursif où $h$ est la hauteur de l'arbre.
On en déduit donc une complexité en $O(h)$ avec $h$ la hauteur de
$T$.

\subsection{Nombre d'appels récursifs (en moyenne)}

Etudions de manière plus attentive la complexité. On se pose la
question suivante :\og Quel est le nombre moyen de segments
$s_j$ dans $T$ tel que la droite passant par $s_j$ intersecte
$[x,y]$?\fg

Posons $s = [x,y]$.

Supposons dans la suite que $s$ n'est inclus dans aucune droite
stockée dans $T$.

Il arrive donc que le segment $s$ intersecte certaines droites.
Utilisons un raisonnement similaire à celui présenté dans
\cite{berg_cheong_kreveld_2008}. On suppose que
l'arbre BSP est construit à l'aide d'auto-partitions
\footnote{
  C'est-à-dire les coupes du plan sont réalisées à partir de
  droites engendrées par un des segments de l'ensemble sur
  base duquel l'arbre est construit}
et de manière aléatoire en supposant chaque permutation
équiprobable. On suppose $T$ construit sur base de
$S = \{s_1,...,s_n\}$. Posons pour $s'$ un segment du plan $l(s')$
la droite passant par les extrémités de $s'$ (elle contient donc
$s'$).

Soit $ i \in \{1,...,n\}$. On définit:
$$ d(s,s_i )= \begin{cases}
  \mbox{\#\{segments entre } s \mbox{ et } s_i
  \mbox{ intersectant } l(s_i) \mbox{\}}
  &\mbox{si } l(s_i) \mbox { intersecte } s \\
  +\infty & \mbox{sinon} \end{cases} $$
Etudions la probabilité que $l(s_i)$ coupe $s$ (notée
$P(l(s_i)$ coupe $s)$), c'est-à-dire lors de la séparation du
plan par $l(s_i)$, $s$ est fragmenté. Supposons que $l(s_i)$ intersecte
$s$ (sinon la probabilité est nulle). On a donc $d(s, s_i)$ finie.

Soit $\sigma  \in S_n $ (l'ensemble des permutations de $n$
éléments, également appelé groupe symétrique d'indice $n$).

S'il existe $k\in\{1, \ldots, n\} $ tel qu'on coupe le plan par
$l(s_k)$ avant de le couper par $l(s_i)$ et que $s$ et $s_i$ se
retrouvent de part et d'autre de $l(s_k)$ alors $l(s_i)$ ne
coupera pas $s$ (car $s_i$ et $s$ ne se trouveront pas dans le
même demi-plan).

En particulier, il faut que les indices des segments situés sur
la droite $l(s_i)$ entre les segments $s$ et $s_i$ apparaissent
après $i$ dans $\sigma$. Il faut donc pour
tous ces segments que leur indice $k$ vérifie
$\sigma(i)<\sigma(k)$ (c'est-à-dire la coupe selon
$l(s_i)$ est effectuée avant celle par $l(s_k)$). Donc $i$ doit
apparaître en premier dans la permutation $\sigma$.

Par conséquent, on a la majoration:
\begin{IEEEeqnarray*}{rcl}
P(l(s_i) \mbox{ coupe s}) &\le&
\frac {\# \mbox {permutations commencant par }
         1 \mbox{ dans } S_{d(s,s_i)+1}}
{\# \mbox {permutations dans } S_{d(s,s_i)+1}} \\
& = & \frac{d(s, s_i)!}{(d(s, s_i) +1)!} =\frac{1}{d(s,s_i)+1}.
\end{IEEEeqnarray*}

Notez que les permutations commençant par 1 sont considérées pour
clarifier les notations, étant donné qu'on considère les
permutations commençant par un élément fixé.

Le nombre moyen de coupes effectuées sur $s$ est donc,
par linéarité de l'espérance:
\begin{IEEEeqnarray*}{lCl}
  \sum_{k=1}^{n}P(\l(s_k)\mbox{ coupe } s)
  & \le &\sum_{k=1}^{n} \frac{1}{d(s,s_k)+1}\\
  & \le & 2 \sum_{k=0}^{\lfloor \frac {n-1}{2} \rfloor}
  \frac 1{k+1}\\
  &\le& 2 \ln \left(\frac{n}{2}\right)
\end{IEEEeqnarray*}

La deuxième inégalité est une conséquence
du fait que pour une distance donnée par rapport à $s$,
il existe au plus deux segments réalisant cette distance.

Nous avons vu dans la section précédente que le coût local est
en temps constant car on ne parcourt $S_T$ que si $T$ est une
feuille par l'hypothèse faite (que $s$ n'est inclus à aucune
droite).

Ce qui nous permet de conclure que le nombre de noeuds où
sont effectués deux appels récursifs est en $O(\ln(n))$. On en
déduit que le nombre de chemins suivis dans l'arbre est en
$O(\ln(n))$. Ceci fournit une complexité totale en
$O(h\times\ln(n))$.