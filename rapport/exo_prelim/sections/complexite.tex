Etudions le coût local dans une feuille. Par définition, $S_T$ contient au plus une donnée.
Donc, le coût local par feuille est en temps constant ($O(1)$).

Intéressons-nous au coût local par noeud interne.
Dans un premier temps, si le segment est contenu dans la droite stockée dans le noeud,
alors on a un coût linéaire en le nombre de fragments dans $S_t$.

Sinon, seules des opérations élémentaires sont effectuées(opérations arithmétiques,
appels de fonctions et comparaisons) d'où le coût local est en $O(1)$.

Etudions le nombre d'appels récursifs. Remarquons d'abord que dans l'algorithme,
chaque noeud est visité au plus une fois.
Par conséquent, Le nombre de noeuds visités maximal correspond au nombre de noeud de noeuds de $T$
. Cela se produit si pour tout appel à $Recherche\_segment(T,x,y)$ , le segment $[x,y]$ intersecte
$D_T$ ou $T$ est une feuille.

Le nombre d'appels récursifs est en $O(n)$ en négligeant les parcours de listes
et où $n$ est le nombre de noeuds de $T$.

Sinon, on a une majoration grossière de la complexité en $O(N \times n)$
où N est le nombre de segments stockés dans $T$.

Etudions de manière plus attentive la complexité. On se pose la question suivante :
Quel est le nombre moyen de segments $s_j$ tel que la droite passant par $s_j$ intersecte $[x,y]$.

Par la référence, on obtient que le nombre moyen d'intersections entre $[x,y]$ et les droites
stockées dans $T$ est en $O(log (n))$. On obtient, par conséquent, que la complexité en moyenne
est en $O(h \times log (n))$ où $h$ est la hauteur de $T$. 