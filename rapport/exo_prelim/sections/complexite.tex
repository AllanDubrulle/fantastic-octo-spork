\subsection{Coûts locaux}
Etudions le coût local dans une feuille.
Par définition, $S_T$ contient au plus une donnée.
Donc, le coût local par feuille est en temps constant ($O(1)$).

Intéressons-nous au coût local par noeud interne.
Dans un premier temps, si le segment est contenu dans
la droite stockée dans le noeud, alors on a un coût linéaire
en le nombre de fragments dans $S_t$ (borné par le nombre de
segments employés dans la construction de l'arbre).

Sinon, seules des opérations élémentaires sont
effectuées(opérations arithmétiques, appels de fonctions
et comparaisons), ce qui entraîne un coût local en $O(1)$.
\subsection{Nombre d'appels récursifs (pire des cas)}
Remarquons d'abord que dans l'algorithme, chaque noeud est
visité au plus une fois.
Par conséquent, Le nombre de noeuds visités maximal correspond
au nombre de noeud de noeuds de $T$.
Il est possible que tous les noeuds de l'arbre soient visités;
si à chaque noeud interne traité, (le fragment considéré du) segment
recherché intersecte la droite correspondant au noeud, cela entraîne
un appel récursif sur chacun des fils du noeud.

Le nombre d'appels récursifs est en $O(N)$ où $N$ correspond
au nombre de noeuds de l'arbre passé en paramètre.

Par conséquent, on en déduit une majoration grossière de
la complexité en $O(N \times n)$ où $n$ est le nombre de
segments stockés dans $T$, étant donné que chaque liste contient
au plus $n$ fragments.

\subsection{Nombre d'appels récursifs (en moyenne)}
Etudions de manière plus attentive la complexité. On se pose la
question suivante :\og Quel est le nombre moyen de segments
$s_j$ dans $T$ tel que la droite passant par $s_j$ intersecte
$[x,y]$?\fg

% D'après \emph{insérer la référence},
% \begin{lem}
% Supposons l'arbre $T$ construit en sélectionnant aléatoirement les
% droites séparant le plan parmi les segments. Alors le nombre moyen
% de fragments est en $O(n\log n)$ où $n$ est le nombre de segments
% contenus dans la scène.
% \end{lem}

%Control-C ; pour commenter/décommenter une région dans Emacs
Par la référence, on obtient que le nombre moyen d'intersections entre $[x,y]$ et les droites
stockées dans $T$ est en $O(log (n))$. On obtient, par conséquent, que la complexité en moyenne
est en $O(h \times log (n))$ où $h$ est la hauteur de $T$.