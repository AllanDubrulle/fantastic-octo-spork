\documentclass[a4paper, 12pt]{article}

\usepackage[utf8]{inputenc}
\usepackage[T1]{fontenc}
\usepackage[francais]{babel}
\usepackage{amsmath, amssymb, amsthm, verbatim}
\usepackage{IEEEtrantools}
\usepackage[margin=1in]{geometry}
\usepackage[colorlinks]{hyperref}
\usepackage{epigraph}
\usepackage{mathrsfs}
\usepackage{algorithm}
\usepackage{algorithmic}

% Définition des théorèmes
\theoremstyle{definition} \newtheorem{lem}{Lemme}
% Traductions des macros pour algo
\renewcommand{\listalgorithmname}{Liste des algorithmes}
\floatname{algorithm}{Algorithme}
\renewcommand{\algorithmicrequire}{\textbf{Entrée:}}
\renewcommand{\algorithmicensure}{\textbf{Sortie:}}
\renewcommand{\algorithmicif}{\textbf{si}}
\renewcommand{\algorithmicthen}{\textbf{alors}}
\renewcommand{\algorithmicelse}{\textbf{sinon}}
\renewcommand{\algorithmicend}{\textbf{fin}}
\renewcommand{\algorithmicreturn}{\textbf{retourner}}

\title{Exercice Préliminaire -- Projet de Structures de Données II}
\author{A. Dubrulle, J. Main}
\date{\today}

\begin{document}
\maketitle

\section{Introduction}
%Document contenant l'introduction de l'exerice
%préliminaire du projet de structures de données II

Nous nous intéressons ici au problème suivant: \og étant donnés
un arbre BSP $T$ représentant une scène dans le plan $\mathbb{R}^2$
et deux points $x$ et $y$ du plan, déterminer si le segment $[x, y]$
est dans l'arbre $T$ \fg.

Posons les notations qui seront utilisées tout au long de la
résolution de l'exercice. On dénote par $T$ à la fois la
racine et l'arbre correspondant.

Si $T$ est une feuille, on dénote par $S_T$ l'ensemble des
(fragments de) segments contenus dans cette feuille.
Notez que $S_T$ contient au plus un élément (par définition de BSP).

Si $T$ est un noeud interne, on note $D$ l'hyperplan affin stocké
en ce noeud. Ce dernier est donné par une équation $f_D(x_1, x_2) = 0$
où $f_D: \mathbb{R}^2 \to \mathbb{R}: (x_1, x_2)\mapsto a x_1 + b x_2 +c$
pour certains $a, b, c\in\mathbb{R}$. Ceci sépare le plan en deux
parties: $$D^+=\{(x_1, x_2)\in\mathbb{R}^2\mid f_D(x_1, x_2) > 0\}$$ et
$$D^-=\{(x_1, x_2)\in\mathbb{R}^2\mid f_D(x_1, x_2) < 0\}$$
Ceci motive la notation $T^-$ (resp. $T^+$) pour le fils gauche
(resp. droit) de $T$ représentant les fragments contenus dans $D^-$
(resp. $D^+$). L'ensemble des segments contenus dans $D$ est noté
$S_T$ de la même manière que pour les feuilles.


\section{Raisonnement mathématique}
% Explication du raisonnement mathématique derrière
% l'algorithme, donné dans une section différente de
% ce dernier

\subsection{Cas de base}
La question présente deux cas de base exclusifs: l'arbre $T$ est
réduit à une feuille ou les points $x, y$ délimitant le segment
recherché sont contenus dans l'hyperplan stocké en la racine de $T$.

Si $T$ est réduit à une feuille, il suffit de tester si le segment
stocké en cette feuille (s'il existe) correspond à $[x, y]$. Si ce
n'est pas le cas ou que la feuille est vide, alors l'algorithme
retourne faux.

Si l'arbre $T$ n'est pas réduit à une feuille et que
les points $x$ et $y$ sont contenus dans $D$, alors par convexité
de $D$, le segment est contenu dans $D$. Le problème se réduit alors
à la recherche d'une donnée au sein d'une liste chaînée.

\subsection{Cas général}
Supposons que $T$ possède deux fils (ie. $T$ n'est pas une feuille).
Discutons les différents cas.

Si $x,y \in D^+$ (resp. $D^-$), alors le segment $[x,y] \subseteq D^+$
(resp. $D^-$) par convexité. Par définition d'arbre BSP,
le segment recherché est contenu dans $T$ si et seulement s'il
est dans $T^+$ (resp. $T^-$). Il suffit donc de poursuivre la
recherche récursivement dans $T^+ $ (resp. $T^-$).

Si $x \in D $ et $y \in D^+$ (resp. $D^-$), alors le segment $[x, y]$ est
contenu dans le demi-plan $D^+$ (resp. $D^-$). Il suffit donc de poursuivre la
recherche récursivement comme précédemment. Le cas $y \in D $ et $x \in D^+$
(resp. $D^-$) est analogue au cas précédent.

Il reste deux cas à considérer: le premier est $ x \in D^+$ et $ y \in D^-$ et
le second est $x \in D^-$ et $ y \in D^+$. Les deux cas étant similaires,
nous supposerons que $ x \in D^+$ et $ y \in D^-$ (les calculs sont
indépendants du cas considéré).
Le segment $[x,y]$ intersecte $D$ en un point qu'on nommera $z$.

Savoir si le segment $[x,y]$ se trouve dans $T$ revient donc à savoir si
les segments $[x,z]$ et $[z,y]$ (le segment est fragmenté) sont
éléments de $T$ et par conséquent de faire deux recherches
récursivement (cfr. cas précédent) et de retourner vrai si et
seulement si les deux appels retournent vrai.

Cependant, il faut connaître $z$ pour appliquer ce raisonnement. Nous allons
donc montrer comment déterminer $z$ à partir de $x$, $y$ et $f_D$.
Posons $g: \mathbb{R}^2 \to \mathbb{R}$ la fonction définie par
$g (x_1, x_2) =  a x_1 + b x_2$ pour tout $(x_1, x_2)\in\mathbb{R}^2$.
Nous avons que $f_D = g + c $ et la fonction $g$ est linéaire.

De plus, nous avons que $f_D(z) = 0$ et $\exists t \in [0,1]$ tel que
$ z = tx+ (1-t)y$
\footnote{En effet, on a par définition:
  $[x, y] = \{tx + (1 - t)y\mid t\in [0,1]\}=\{ty + (1 - t)x\mid t\in [0,1]\}$}.
Donc, par linéarité de $g$:
\begin{equation}
  0 = f_D(z)  = f_D (tx+ (1-t)y) = t(g(x)-g(y))+g(y)+c
\end{equation}

Par conséquent,
\begin{equation} \label{eq:t}
    t = \frac{(-c - g(y))}{(g(x)-g(y))} = \frac{-(f_D(y)+2c)}{f_D(x)-f_D(y)}
\end{equation}
L'expression de $t$ est bien définie car  $g(x) \neq g(y)$ car
$ g(y) < -c < g(x)$.


\section{Algorithme}
% Document pour donner l'algorithme de recherche
% de segment dans un BSP donné.
% A input dans le fichier principal.

\begin{algorithm}
  \caption{Recherche\_segment($T, x, y$)}
  \begin{algorithmic}[1]
    \REQUIRE Un arbre BSP $T$, deux points $x$ et $y$ du plan réel.
    \ENSURE Vrai si et seulement si le segment $[x, y]$ est contenu dans $T$.
    \IF{$T$ est réduit à une feuille}
    \RETURN $[x, y]\in S_T$
    \ENDIF
    \IF{$f_D(x) > 0$}
      \IF{$f_D(y) \geq 0$}
      \RETURN Recherche\_segment($T^+$, $x$, $y$)
      \ELSE
      \STATE $z \leftarrow D \cap [x, y]$
      \RETURN Recherche\_segment($T^+$, $x$, $z$) $\land$ Recherche\_segment($T^-$, $z$, $y$)
      \ENDIF
      \ELSE
      \IF{$f_D(x) = f_D(y) = 0$}
      \RETURN $[x, y]\in S_T$
      \ELSIF{$f_D(y) \leq 0$}
      \RETURN Recherche\_segment($T^-$, $x$, $y$)
      \ELSE
      \STATE $z \leftarrow D \cap [x, y]$
      \RETURN Recherche\_segment($T^-$, $x$, $z$) $\land$ Recherche\_segment($T^+$, $z$, $y$)
      \ENDIF
    \ENDIF
  \end{algorithmic}
\end{algorithm}

\section{Complexité}
\subsection{Coûts locaux}
Etudions le coût local dans une feuille.
Par définition, $S_T$ contient au plus une donnée.
Donc, le coût local par feuille est en temps constant ($O(1)$).

Intéressons-nous au coût local par noeud interne.
Dans un premier temps, si le segment est contenu dans
la droite stockée dans le noeud, alors on a un coût linéaire
en le nombre de fragments dans $S_t$ (borné par le nombre de
segments employés dans la construction de l'arbre).

Sinon, seules des opérations élémentaires sont
effectuées(opérations arithmétiques, appels de fonctions
et comparaisons), ce qui entraîne un coût local en $O(1)$.

\subsection{Nombre d'appels récursifs (pire des cas)}
Remarquons d'abord que dans l'algorithme, chaque noeud est
visité au plus une fois.
Par conséquent, Le nombre de noeuds visités maximal correspond
au nombre de noeud de noeuds de $T$.
Il est possible que tous les noeuds de l'arbre soient visités;
si à chaque noeud interne traité, (le fragment considéré du) segment
recherché intersecte la droite correspondant au noeud, cela entraîne
un appel récursif sur chacun des fils du noeud.
Le nombre d'appels récursifs est en $O(N)$ où $N$ correspond
au nombre de noeuds de l'arbre passé en paramètre.

Par conséquent, on en déduit une majoration grossière de
la complexité en $O(N \times n)$ où $n$ est le nombre de
segments stockés dans $T$, étant donné que chaque liste contient
au plus $n$ fragments.

\subsection{Nombre d'appels récursifs (cas simple)}
$[x,y]$ n'intersecte aucune droite.
L'algorithme suit un chemin de l'arbre. Nous avons donc
au maximum $h$ appels récursif où $h$ est la hauteur de l'arbre.
On trouve donc une complexité en ($O(h)$) avec h la hauteur de $T$

\subsection{Nombre d'appels récursifs (en moyenne)}

Etudions de manière plus attentive la complexité. On se pose la
question suivante :\og Quel est le nombre moyen de segments
$s_j$ dans $T$ tel que la droite passant par $s_j$ intersecte
$[x,y]$?\fg

% D'après \emph{insérer la référence},
% \begin{lem}
% Supposons l'arbre $T$ construit en sélectionnant aléatoirement les
% droites séparant le plan parmi les segments. Alors le nombre moyen
% de fragments est en $O(n\log n)$ où $n$ est le nombre de segments
% contenus dans la scène.
% \end{lem}

Posons $s = [x,y]$.

Situation 1 : $s$ n'est inclus dans aucune droite stockées dans $T$.

Plus explicitement, $s$ intersecte certaines droites. Utilisons un raisonnement similaire
au livre. On suppose que l'arbre BSP est construit à l'aide d'auto-partitions et de
manière aléatoire en supposant chaque permutation équiprobable.
On suppose $T$ construit sur base de $S = {s_1,...,s_n}$. Soit $ i \in \{1,...,n\}$.
Posons :

$l(s_i) =$ la droite stockée dans $T$ correspondant à $S_i$ et
$$ d(s,s_i )= \begin{cases} \mbox{\#\{segments entre } s \mbox{ et } s_i
\mbox{ intersectant } l(s_i) \mbox{\}}
&\mbox{si } l(s_i) \mbox { intersecte } s \\ +\infty & \mbox{sinon} \end{cases} $$
Etudions la probabilité que $l(s_i)$ coupe $s$ (notée $P(l(s_i)$ coupe $s)$).
Soit $\sigma  \in S_n $( l'ensemble des permutations à $n$ éléments).

Si $ \exists 1 \le  k \le n $ tel qu'on coupe le plan par $s_k$ avant de couper par $s_i$ et que $s$ et $s_j$ se retrouve chacun d'un côté de $l(s_k)$ alors $s_j$ ne coupe pas $s$.
Il faut donc considerer que $\sigma (j) <= \sigma (k)$ pour tous les $k$ où cela arrive.
En particulier, cela arrive pour tout segment entre $s$ et $s_n$ intersectant $l(s_n)$.
Il faut donc que pour toutes permutations de $\{ s, s_1,...,s_{d(s,s_i)} \}$ , $s_i$ soit le premier.
Par conséquent, 

$P(l(s_i)$ coupe $s) \le
\frac {\# \mbox {permutations commencant par } j \mbox{ dans } S_{d(s,s_i)+1}}
{\# \mbox {permutations dans } S_{d(s,s_i)+1}} = \frac 1 {d(s,s_i)+1}$.

On a donc en moyenne. $$\sum_{k=1}^{n} \frac 1 {d(s,s_i)+1}        \le 2 \sum_{k=0}^{\lfloor \frac {n-1} 2 \rfloor}  \frac 1{k+1} \le 2 ln (n/2) $$


Nous avons donc vu dans la section précédente que le coût local est en ($O(1)$) car on ne parcourt $S_T$ que si $T$ est une feuille par l'hypothèse faite.
Ce qui nous permet de conclure par la formule utilisée au cours que la complexité
de l'algorithme est en ($O(ln(n)$)





\end{document}
