\documentclass[a4paper, 12pt]{article}

\usepackage[utf8]{inputenc}
\usepackage[T1]{fontenc}
\usepackage[francais]{babel}
\usepackage{amsmath, amssymb, amsthm, verbatim}
\usepackage{IEEEtrantools}
\usepackage[margin=1in]{geometry}
\usepackage[colorlinks]{hyperref}
\usepackage{epigraph}
\usepackage{mathrsfs}
\usepackage{algorithm, algorithmic}

% Traductions des macros pour algo
\renewcommand{\algorithmicrequire}{\textbf{Entrée:}}
\renewcommand{\algorithmicensure}{\textbf{Sortie:}}
\renewcommand{\algorithmicif}{\textbf{si}}
\renewcommand{\algorithmicthen}{\textbf{alors}}
\renewcommand{\algorithmicelse}{\textbf{sinon}}
\renewcommand{\algorithmicend}{\textbf{fin}}
\renewcommand{\algorithmicreturn}{\textbf{retourner}}

\title{Exercice Préliminaire -- Projet de Structures de Données II}
\author{A. Dubrulle, J. Main}
\date{\today}

\begin{document}
\maketitle

\section{Introduction}
%Document contenant l'introduction de l'exerice
%préliminaire du projet de structures de données II

Nous nous intéressons ici au problème suivant: \og étant donnés
un arbre BSP $T$ représentant une scène dans le plan $\mathbb{R}^2$
et deux points $x$ et $y$ du plan, déterminer si le segment $[x, y]$
est dans l'arbre $T$ \fg.

Posons les notations qui seront utilisées tout au long de la
résolution de l'exercice. On dénote par $T$ à la fois la
racine et l'arbre correspondant.

Si $T$ est une feuille, on dénote par $S_T$ l'ensemble des
(fragments de) segments contenus dans cette feuille.
Notez que $S_T$ contient au plus un élément (par définition de BSP).

Si $T$ est un noeud interne, on note $D$ l'hyperplan affin stocké
en ce noeud. Ce dernier est donné par une équation $f_D(x_1, x_2) = 0$
où $f_D: \mathbb{R}^2 \to \mathbb{R}: (x_1, x_2)\mapsto a x_1 + b x_2 +c$
pour certains $a, b, c\in\mathbb{R}$. Ceci sépare le plan en deux
parties: $$D^+=\{(x_1, x_2)\in\mathbb{R}^2\mid f_D(x_1, x_2) > 0\}$$ et
$$D^-=\{(x_1, x_2)\in\mathbb{R}^2\mid f_D(x_1, x_2) < 0\}$$
Ceci motive la notation $T^-$ (resp. $T^+$) pour le fils gauche
(resp. droit) de $T$ représentant les fragments contenus dans $D^-$
(resp. $D^+$). L'ensemble des segments contenus dans $D$ est noté
$S_T$ de la même manière que pour les feuilles.


\section{Raisonnement mathématique}
% Explication du raisonnement mathématique derrière
% l'algorithme, donné dans une section différente de
% ce dernier

\subsection{Cas de base}
La question présente deux cas de base exclusifs: l'arbre $T$ est
réduit à une feuille ou les points $x, y$ délimitant le segment
recherché sont contenus dans l'hyperplan stocké en la racine de $T$.

Si $T$ est réduit à une feuille, il suffit de tester si le segment
stocké en cette feuille (s'il existe) correspond à $[x, y]$. Si ce
n'est pas le cas ou que la feuille est vide, alors l'algorithme
retourne faux.

Si l'arbre $T$ n'est pas réduit à une feuille et que
les points $x$ et $y$ sont contenus dans $D$, alors par convexité
de $D$, le segment est contenu dans $D$. Le problème se réduit alors
à la recherche d'une donnée au sein d'une liste chaînée.

\subsection{Cas général}
Supposons que $T$ possède deux fils (ie. $T$ n'est pas une feuille).
Discutons les différents cas.

Si $x,y \in D^+$ (resp. $D^-$), alors le segment $[x,y] \subseteq D^+$
(resp. $D^-$) par convexité. Par définition d'arbre BSP,
le segment recherché est contenu dans $T$ si et seulement s'il
est dans $T^+$ (resp. $T^-$). Il suffit donc de poursuivre la
recherche récursivement dans $T^+ $ (resp. $T^-$).

Si $x \in D $ et $y \in D^+$ (resp. $D^-$), alors le segment $[x, y]$ est
contenu dans le demi-plan $D^+$ (resp. $D^-$). Il suffit donc de poursuivre la
recherche récursivement comme précédemment. Le cas $y \in D $ et $x \in D^+$
(resp. $D^-$) est analogue au cas précédent.

Il reste deux cas à considérer: le premier est $ x \in D^+$ et $ y \in D^-$ et
le second est $x \in D^-$ et $ y \in D^+$. Les deux cas étant similaires,
nous supposerons que $ x \in D^+$ et $ y \in D^-$ (les calculs sont
indépendants du cas considéré).
Le segment $[x,y]$ intersecte $D$ en un point qu'on nommera $z$.

Savoir si le segment $[x,y]$ se trouve dans $T$ revient donc à savoir si
les segments $[x,z]$ et $[z,y]$ (le segment est fragmenté) sont
éléments de $T$ et par conséquent de faire deux recherches
récursivement (cfr. cas précédent) et de retourner vrai si et
seulement si les deux appels retournent vrai.

Cependant, il faut connaître $z$ pour appliquer ce raisonnement. Nous allons
donc montrer comment déterminer $z$ à partir de $x$, $y$ et $f_D$.
Posons $g: \mathbb{R}^2 \to \mathbb{R}$ la fonction définie par
$g (x_1, x_2) =  a x_1 + b x_2$ pour tout $(x_1, x_2)\in\mathbb{R}^2$.
Nous avons que $f_D = g + c $ et la fonction $g$ est linéaire.

De plus, nous avons que $f_D(z) = 0$ et $\exists t \in [0,1]$ tel que
$ z = tx+ (1-t)y$
\footnote{En effet, on a par définition:
  $[x, y] = \{tx + (1 - t)y\mid t\in [0,1]\}=\{ty + (1 - t)x\mid t\in [0,1]\}$}.
Donc, par linéarité de $g$:
\begin{equation}
  0 = f_D(z)  = f_D (tx+ (1-t)y) = t(g(x)-g(y))+g(y)+c
\end{equation}

Par conséquent,
\begin{equation} \label{eq:t}
    t = \frac{(-c - g(y))}{(g(x)-g(y))} = \frac{-(f_D(y)+2c)}{f_D(x)-f_D(y)}
\end{equation}
L'expression de $t$ est bien définie car  $g(x) \neq g(y)$ car
$ g(y) < -c < g(x)$.


\section{Algorithme}
% Document pour donner l'algorithme de recherche
% de segment dans un BSP donné.
% A input dans le fichier principal.

\begin{algorithm}
  \caption{Recherche\_segment($T, x, y$)}
  \begin{algorithmic}[1]
    \REQUIRE Un arbre BSP $T$, deux points $x$ et $y$ du plan réel.
    \ENSURE Vrai si et seulement si le segment $[x, y]$ est contenu dans $T$.
    \IF{$T$ est réduit à une feuille}
    \RETURN $[x, y]\in S_T$
    \ENDIF
    \IF{$f_D(x) > 0$}
      \IF{$f_D(y) \geq 0$}
      \RETURN Recherche\_segment($T^+$, $x$, $y$)
      \ELSE
      \STATE $z \leftarrow D \cap [x, y]$
      \RETURN Recherche\_segment($T^+$, $x$, $z$) $\land$ Recherche\_segment($T^-$, $z$, $y$)
      \ENDIF
      \ELSE
      \IF{$f_D(x) = f_D(y) = 0$}
      \RETURN $[x, y]\in S_T$
      \ELSIF{$f_D(y) \leq 0$}
      \RETURN Recherche\_segment($T^-$, $x$, $y$)
      \ELSE
      \STATE $z \leftarrow D \cap [x, y]$
      \RETURN Recherche\_segment($T^-$, $x$, $z$) $\land$ Recherche\_segment($T^+$, $z$, $y$)
      \ENDIF
    \ENDIF
  \end{algorithmic}
\end{algorithm}

\section{Complexité}
%TODO donner la complexité de l'algorithme et discuter les cas
%avec des subsections
\end{document}
