\documentclass[a4paper, 12pt]{article}

\usepackage[utf8]{inputenc}
\usepackage[T1]{fontenc}
\usepackage[francais]{babel}
\usepackage{amsmath, amssymb, amsthm, verbatim}
\usepackage{IEEEtrantools}
\usepackage[margin=1in]{geometry}
\usepackage[colorlinks]{hyperref}
\usepackage{epigraph}
\usepackage{mathrsfs}
\usepackage{algorithm}
\usepackage{algorithmic}

% Définition des théorèmes
\theoremstyle{definition} \newtheorem{lem}{Lemme}
% Traductions des macros pour algo
\renewcommand{\listalgorithmname}{Liste des algorithmes}
\floatname{algorithm}{Algorithme}
\renewcommand{\algorithmicrequire}{\textbf{Entrée:}}
\renewcommand{\algorithmicensure}{\textbf{Sortie:}}
\renewcommand{\algorithmicif}{\textbf{si}}
\renewcommand{\algorithmicthen}{\textbf{alors}}
\renewcommand{\algorithmicelse}{\textbf{sinon}}
\renewcommand{\algorithmicend}{\textbf{fin}}
\renewcommand{\algorithmicreturn}{\textbf{retourner}}

\title{Exercice Préliminaire -- Projet de Structures de Données II}
\author{A. Dubrulle, J. Main}
\date{\today}

\begin{document}
\maketitle

\section{Introduction}
% contexte
Le présent document se concentre sur un problème d'affichage graphique;
étant donné un point de vue, il s'agit de représenter ce qu'il voit en
prenant en compte le fait que certains objets en cachent d'autres.
L'objectif est d'afficher à partir d'un ensemble de segments dans le
plan une représentation en une dimension de ce que voit le point de vue
(voir les figures
\ref{intro:ill}, \ref{intro2:ill}).

L'algorithme du peintre est la solution employée pour résoudre le problème et
la structure de données associée, l'arbre BSP
(\emph{binary space partition tree}), est étudiée dans ce document.

Une implémentation en Java des arbres BSP et de l'algorithme
du peintre a été réalisée, grâce à laquelle plusieurs
heuristiques de construction d'arbres BSP ont pu être testées et
comparées à l'aide de tests sur machine.

Le mode d'emploi de l'application est également fourni dans ce rapport.

\begin{figure}[!h]
  \begin{center}
    \caption{Visualisation d'une scène par un \oe{}il}%
    \label{intro:ill}
    \begin{tikzpicture}[scale=1, thick]
      \begin{scope}[rotate=90]
        \draw [dotted] (45:4) -- (0, 0) -- (-45:4);
        \draw [thin, dotted, domain=-45:45] plot ({4*cos(\x)}, {4*sin(\x)});
        \draw [green] (50:3) -- (35:2);
        \draw [green, dotted] (45:4) -- (0, 0);
        \draw [green, dotted] (35:4) -- (0, 0);
        \draw [green, domain=35:45] plot ({4*cos(\x)}, {4*sin(\x)});
        \draw [blue] (20:3) -- (5:3);
        \draw [blue, dotted] (20:4) -- (0, 0);
        \draw [blue, dotted] (5:4) -- (0, 0);
        \draw [blue, domain=5:20] plot ({4*cos(\x)}, {4*sin(\x)});
        \draw [red] (0:1) -- (-25:2);
        \draw [red, dotted] (0:4) -- (0, 0);
        \draw [red, dotted] (-25:4) -- (0, 0);
        \draw [red, domain=-25:0] plot ({4*cos(\x)}, {4*sin(\x)});
        \draw [purple] (-20:3) -- (-40:2);
        \draw [purple, dotted] (-40:4) -- (0, 0);
        \draw [purple, domain=-25:-40] plot ({4*cos(\x)}, {4*sin(\x)});
        \draw (45:1) -- (0, 0) -- (-45:1);
        \fill [fill=black] (0, 0) -- (45:0.5) -- (-45:0.5) -- cycle;
      \end{scope}
    \end{tikzpicture}
\end{center}
\end{figure}
\begin{figure}[!h]
  \begin{center}
    \caption{Représentation de la vision de la figure \ref{intro:ill}}%
    \label{intro2:ill}
    \begin{tikzpicture}[thick, scale=0.1]
      \draw [dotted, thin] (0, 0) -- (90, 0);
      \draw [green] (0, 0) -- (10, 0);
      \draw [blue] (25, 0) -- (40, 0);
      \draw [red] (45, 0) -- (70, 0);
      \draw [purple] (70, 0) -- (85, 0);
    \end{tikzpicture}
\end{center}
\end{figure}
%%% Local Variables:
%%% mode: latex
%%% TeX-master: "../rapportGp1"
%%% End:


%%% Local Variables:
%%% mode: latex
%%% TeX-master: "../rapportGp1"
%%% End:


\section{Raisonnement mathématique}
% Explication du raisonnement mathématique derrière
% l'algorithme, donné dans une section différente de
% ce dernier

\subsection{Cas de base}
La question présente deux cas de base exclusifs: l'arbre $T$ est
réduit à une feuille ou les points $x, y$ délimitant le segment
recherché sont contenus dans l'hyperplan stocké en la racine de $T$.

Si $T$ est réduit à une feuille, il suffit de tester si le segment
stocké en cette feuille (s'il existe) correspond à $[x, y]$. Si ce
n'est pas le cas ou que la feuille est vide, alors l'algorithme
retourne faux.

Si les points $x$ et $y$ sont contenus dans $D$, alors par convexité
de $D$, le segment est contenu dans $D$. Le problème se réduit alors
à la recherche d'une donnée au sein d'une liste chaînée.

\subsection{Cas général}
Supposons que $T$ possède deux fils (ie. $T$ n'est pas une feuille).
Discutons les différents cas.


\section{Algorithme}
% Document pour donner l'algorithme de recherche
% de segment dans un BSP donné.
% A input dans le fichier principal.

\begin{algorithm}
  \caption{Recherche\_segment($T, x, y$)}
  \begin{algorithmic}[1] \label{algo:main}
    \REQUIRE Un arbre BSP $T$, deux points $x$ et $y$ du plan.
    \ENSURE Vrai si et seulement si le segment $[x, y]$ est contenu dans $T$.
    \IF{$T$ est réduit à une feuille}
    \RETURN $[x, y]\in S_T$
    \ENDIF
    \IF{$f_D(x) > 0$}
      \IF{$f_D(y) \geq 0$}
      \RETURN Recherche\_segment($T^+$, $x$, $y$)
      \ELSE
      \STATE $z \leftarrow D \cap [x, y]$
      \RETURN Recherche\_segment($T^+$, $x$, $z$) $\land$
              Recherche\_segment($T^-$, $z$, $y$)
      \ENDIF
      \ELSE
      \IF{$f_D(x) = f_D(y) = 0$}
      \RETURN $[x, y]\in S_T$
      \ELSIF{$f_D(y) \leq 0$}
      \RETURN Recherche\_segment($T^-$, $x$, $y$)
      \ELSE
      \STATE $z \leftarrow D \cap [x, y]$
      \RETURN Recherche\_segment($T^-$, $x$, $z$) $\land$
              Recherche\_segment($T^+$, $z$, $y$)
      \ENDIF
    \ENDIF
  \end{algorithmic}
\end{algorithm}

La vérification de la présence d'un segment donné est détaillé à
l'algorithme \ref{algo:main}.

Les détails suivants sont omis afin de ne pas alourdir l'algorithme:
l'appartenance dans les ensembles stockés en $T$ (ensembles notés
$S_T$) et le calcul de l'intersection du segment et de la
droite stockée en $T$.

Afin de vérifier l'appartenance d'un segment à un ensemble, il
suffit de parcourir l'ensemble, représenté ici par une liste
chaînée, et de tester l'égalité entre les éléments et le segment
en entrée.

Il suffit donc d'expliciter l'égalité de segments et d'appliquer
un algorithme de recherche classique dans une liste chaînée.
Deux segments sont égaux si et seulement s'ils ont les
mêmes extrémités, c'est-à-dire:
$$ [x_1, y_1] = [x_2, y_2] \iff
(x_1 = x_2 \land y_1 = y_2) \lor (x_1 = y_2 \land y_1 = x_2) $$

Il reste à discuter le calcul de l'intersection d'un segment
et d'une droite dans le cas où elle n'est pas vide. Or, ces
considérations sont détaillées dans le raisonnement mathématique
(se référer à l'égalité (\ref{eq:t}) et se rappeler que $z = tx + (1-t) y$).


\section{Complexité}
\subsection{Coût local}

Etudions le coût local dans une feuille. Par définition, $S_T$ contient au plus une donnée.
Donc, le coût local par feuille est en temps constant ($O(1)$).

Intéressons-nous au coût local par noeud interne.
Dans un premier temps, si le segment est contenu dans la droite stockée dans le noeud,
alors on a un coût linéaire en le nombre de fragments dans $S_T$.

Sinon, seules des opérations élémentaires sont effectuées(opérations arithmétiques,
appels de fonctions et comparaisons) d'où le coût local est en $O(1)$.

\subsection{Appels récursifs}

Intéressons nous au nombre d'appels récursifs de l'algorithme.
Posons $s = [x,y]$
Situation 1 : $s$ n'est inclus dans aucune droite stockées dans $T$.

Cas 1 : le pire des cas.
Remarquons d'abord que dans l'algorithme, chaque noeud est visité au plus une fois.
Par conséquent, Le nombre de noeuds visités maximal correspond au nombre de noeuds de $T$.
Cela se produit si pour tout appel à $Recherche\_segment(T,x,y)$ , le segment $s$ intersecte
$D_T$ ou $T$ est une feuille. nous avons donc $n$ appels récursifs 

Cas 2 :  $s$ n'intersecte aucune droite.
L'algorithme suit un chemin de l'arbre. Nous avons donc
au maximum $h$ appels récursif où $h$ est la hauteur de l'arbre.

Cas 3 : Cas moyen. 
Plus explicitement, $s$ intersecte certaines droites. Utilisons un raisonnement similaire
au livre. On suppose que l'arbre BSP est construit à l'aide d'auto-partitions et de
manière aléatoire en supposant chaque permutation équiprobable.
On suppose $T$ construit sur base de $S = {s_1,...,s_n}$. Soit $ i \in \{1,...,n\}$.
Posons :

$l(s_i) =$ la droite stockée dans $T$ correspondant à $S_i$ et
$$ d(s,s_i )= \begin{cases} \mbox{\#\{segments entre } s \mbox{ et } s_i
\mbox{ intersectant } l(s_i) \mbox{\}}
&\mbox{si } l(s_i) \mbox { intersecte } s \\ +\infty & \mbox{sinon} \end{cases} $$
Etudions la probabilité que $l(s_i)$ coupe $s$ (notée $P(l(s_i)$ coupe $s)$).
Soit $\sigma  \in S_n $( l'ensemble des permutations à $n$ éléments).

Si $ \exists 1 \le  k \le n $ tel qu'on coupe le plan par $s_k$ avant de couper par $s_i$ et que $s$ et $s_j$ se retrouve chacun d'un côté de $l(s_k)$ alors $s_j$ ne coupe pas $s$.
Il faut donc considerer que $\sigma (j) <= \sigma (k)$ pour tous les $k$ où cela arrive.
En particulier, cela arrive pour tout segment entre $s$ et $s_n$ intersectant $l(s_n)$.
Il faut donc que pour toutes permutations de $\{ s, s_1,...,s_{d(s,s_i)} \}$ , $s_i$ soit le premier.
Par conséquent, 

$P(l(s_i)$ coupe $s) \le
\frac {\# \mbox {permutations commencant par } j \mbox{ dans } S_{d(s,s_i)+1}}
{\# \mbox {permutations dans } S_{d(s,s_i)+1}} = \frac 1 {d(s,s_i)+1}$.

On a donc en moyenne. $$\sum_{k=1}^{n} \frac 1 {d(s,s_i)+1}        \le 2 \sum_{k=0}^{\lfloor \frac {n-1} 2 \rfloor}  \frac 1{k+1} \le 2 ln (n/2) $$


Nous avons donc vu dans la section précédente que le coût local est en ($O(1)$) car on ne parcourt $S_T$ que si $T$ est une feuille par l'hypothèse faite.
Ce qui nous permet de conclure par la formule utilisée au cours que la complexité
de l'algorithme est :

-Cas 1: en ($O(n)$) avec n le nombre de noeuds de $T$ 

-Cas 2: en ($O(h)$) avec h la hauteur de $T$

-Cas 3: en ($O(ln(n)$)



Ancien texte : 
(*Sinon, on a une majoration grossière de la complexité en $O(N \times n)$
où N est le nombre de segments stockés dans $T$.*)

(*Etudions de manière plus attentive la complexité. On se pose la question suivante :
Quel est le nombre moyen de segments $s_j$ tel que la droite passant par $s_j$ intersecte $[x,y]$.

Par la référence, on obtient que le nombre moyen d'intersections entre $[x,y]$ et les droites
stockées dans $T$ est en $O(log (n))$. On obtient, par conséquent, que la complexité en moyenne
est en $O(h \times log (n))$ où $h$ est la hauteur de $T$. *) 

\end{document}
